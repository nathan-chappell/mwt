\chapter{Proof of the \MWT}

\section{Every V-Cone is an H-Cone}

\begin{Def}[Coordinate Projection]{
		Let $I$ be the identity matrix.  Then the matrix $I'$ formed by deleting some rows from $I$ is called a \textbf{coordinate-projection}.
	}\end{Def}

The proof rests on the following two propostions:
\begin{itemize}
	\item[\Vlift] Every V-Cone is a coordinate-projection of an H-Cone.
	\item[\Vproj] Every coordinate-projection of an H-Cone is an H-Cone.
\end{itemize}
\begin{proof}
	Given {\Vlift} and {\Vproj}, the proof follows simply.  Given a V-Cone, we use {\Vlift}, to get a description involving coordinate-projection of an H-Cone.  Then we can apply {\Vproj} in order to get an H-Cone.
\end{proof}

\begin{proof}[Proof of {\Vlift}]
	We prove that every V-Cone is a coordinate-projection of an H-Cone, by giving an explicit formula.  Let ${\mU}$, and observe that
	\[ \cone(U) = \set{U\t \st \t \in \R^{\Udim},\, \t \geq \0} =
		\set{\xv \st (\exists \tv)\,\x = U\t,\, \t \geq \0} \]
	We will collect $\t$ and $\x$ on the left side of the inequality, treating $\t$ as a variable and expressing its contraints as linear inequalities, then project away the coordinates corresponding to $\t$.  The following expression takes one step:
	\begin{equation}\label{eq:tleqz}
		\t \geq \0 \Leftrightarrow -I\t \leq \0
	\end{equation}
	And using the equality: $a = 0 \Leftrightarrow a \leq 0 \land -a \leq 0$, and block matrix notation, we take the second step.
	\begin{equation}\label{eq:xeqt}
		\x = U\t \Leftrightarrow \x - U\t = \0 \Leftrightarrow
		\begin{pmatrix*}[r] I & -U \\ -I & U \end{pmatrix*} \xt \leq \0
	\end{equation}
	Comparing \eqref{eq:tleqz} and \eqref{eq:xeqt}, we define a new matrix $A' \in \R^{(\Udim+2d)\times(d+\Udim)}$:
	\begin{equation}\label{eq:Alift}
		A' = \begin{pmatrix*}[r] \0 & -I \\ I & -U \\ -I & U \end{pmatrix*}
	\end{equation}
	then we can rewrite $\cone(U)$:
	\begin{equation*}
		\cone(U) = \set{ \xv \St A'\xt \leq \0}
	\end{equation*}
	Let $\Pi \in \set{0,1}^{d\times(d+\Udim)}$ be the identity matrix in $\R^{(d+\Udim)\times(d+\Udim)}$, but with the last $\Udim$-rows deleted.  Then $\Pi$ is a coordinate projection, and the above expression can be written:
	\begin{equation}\label{eq:vconelift}
		\cone(U) = \Pi\left(\set{ \y \in \R^{d+\Udim} \st A'\y \leq \0}\right)
	\end{equation}
	This is a coordinate projection of an H-Cone, and {\Vlift} is shown.
\end{proof}
To prove {\Vproj}, we use two separate propositions.
\begin{Prop}{\label{prop:hconezero}
		Let $B\in\R^{m'\times(d+\Udim)}$, $B'$ be $B$ with the last $\Udim$ columns deleted, and $\Pi$ the identity matrix with the last $\Udim$ rows deleted (i.e. $B' = \Pi B$).  Furthermore, suppose that the last $\Udim$ columns of $B$ are $\0$.  Then
		\[ \Pi\left(\set{\y \in \R^{d+\Udim} \st B\y \leq \0}\right) =
			\set{\x\in\R^d\st B'\x\leq\0} \]
	}\end{Prop}
\begin{proof}
	Recall that $B\y \leq \0$ means that $(\forall i)\ip{B_i}{\y} \leq 0$.  Because the last $\Udim$ columns of $B$ are $\0$, any row $B_i$ of $B$ can be written $(B'_i,\0)$, with $\0 \in \R^\Udim$.  We can also rewrite $\y\in\R^{d+\Udim}$ as $(\x,\w)$ with $\x\in\R^d,\w\in\R^\Udim$, so that $\x = \Pi(\y)$.  Then
	\[ \ip{B}{\y} = \ip{(B'_i,\0)}{(\x,\w)} = \ip{B'_i}{\x} = \ip{B'_i}{\Pi(\y)} \]
	It follows that
	\[ \ip{B_i}{\y} \leq 0 \Leftrightarrow \ip{B'_i}{\Pi(\y)} \leq 0 \]
	Since $B_i$ is an arbitrary row of $B$, the proposition is shown.
\end{proof}

In order to use the above proposition, we need a matrix with columns which are $\0$.  The next proposition shows us how to obtain such a matrix from another, while maintaining certain properties.

\begin{Prop}{\label{prop:hconeproj}
		Let $\mB$, $1 \leq k \leq (d+p)$, and $\x = \sum_{i\neq k}x_i \e_i$.  Then there exists a matrix $\mC$ with the following properties:
		\begin{enumerate}
			\item Every row of $B'$ is a postive linear combination of rows of $B$.
			\item $m_2$ is finite.
			\item The $k$-th column of $B'$ is $\0$.
			\item \((\exists t)B(\x + t\e_k) \leq \0 \Leftrightarrow B'\x \leq \0\)
		\end{enumerate}
	}\end{Prop}
\begin{proof}
	Partition the rows of $B$ as follows:
	\begin{alignat*}{3}
		 & P &  & = i & \st & \Bik > 0 \\
		 & N &  & = j & \st & \Bjk < 0 \\
		 & Z &  & = l & \st & \Blk = 0
	\end{alignat*}
	Then let $B'$ be a matrix with rows of the following forms:
	\begin{alignat*}{3}
		 & C_l    &  & = B_l                 & \st & l \in Z          \\
		 & C_{ij} &  & = \Bik B_j - \Bjk B_i & \st & i \in P, j \in N
	\end{alignat*}
	\textit{1} and \textit{2} are clear.  \textit{3} can be seen from:
	\begin{align}
		 & \ip{C_l}{\e_k} = 0 \nonumber                                                                        \\
		 & \ip{C_{ij}}{\e_k} = \ip{\Bik B_j - \Bjk B_i}{\e_k} = \Bik \Bjk - \Bjk \Bik = 0\label{eq:vdropZrows}
	\end{align}
	The right direction of \textit{4} is shown in the following calculations.  Because $\Blk = 0$:
	\[ \ip{B_l}{\x+t\e_k} = \ip{B_l}{\x} + t\Blk = \ip{B_l}{\x} = \ip{C_l}{\x} \]
	So:
	\[ \ip{B_l}{\x+t\e_k} \leq 0 \Rightarrow \ip{C_l}{\x} \leq 0 \]
	For rows indexed by $P,N$, we observe \eqref{eq:vdropZrows}, and have:
	\[ \ip{\Bik B_j - \Bjk B_i}{\x + t\e_k} = \ip{\Bik B_j - \Bjk B_i}{\x} \]
	Now, we use property \textit{1}:
	\[ \ip{B_i}{\x+t\e_k} \leq 0,\; \ip{B_j}{\x+t\e_k} \leq 0 \Rightarrow
		\ip{\Bik B_j - \Bjk B_i}{\x + t\e_k} \leq 0\]
	Therefore
	\[ \ip{\Bik B_j - \Bjk B_i}{\x} \leq 0 \]

	Now suppose that $B'\x \leq \0$.  The task is to find a $t$ so that $B\x \leq \0$.  Looking at \eqref{eq:vdropZrows}, any choice of $t$ we make will be okay for rows indexed by $Z$.  So the task is to find a $t$ so that the inequality holds for rows indexed by $P$ and $N$.  Observe
	\begin{alignat*}{3}
		\faij & \quad & \ip{\Bik B_j - \Bjk B_i}{\x}    & \leq  \; 0                                & \Leftrightarrow \\
		\faij & \quad & \ip{\Bik B_j}{\x}               & \leq  \; \ip{\Bjk B_i}{\x}                & \Leftrightarrow \\
		\faij & \quad & \ip{B_i/\Bik}{\x}               & \leq  \; \ip{B_j/\Bjk}{\x}                & \Leftrightarrow \\
		      & \quad & \max_{i\in P} \ip{B_i/\Bik}{\x} & \leq  \; \min_{j\in N}  \ip{B_j/\Bjk}{\x} &
	\end{alignat*}
	Note that the third inequality changes directions because $\Bjk < 0$.  Now we choose $t$ to lie in this last interval, and show that we can use it to satisfy all of the constraints given by $ B$.  So, we have a $t$ such that
	\[ \max_{i\in P} \ip{B_i/\Bik}{\x} \leq t \leq \min_{j\in N} \ip{B_j/\Bjk}{\x} \]
	In particular,
	\begin{alignat*}{2}
		(\forall j\in N)\quad & \ip{B_j/\Bjk}{\x}     \; & \geq & \; t \Rightarrow \\
		(\forall j\in N)\quad & \ip{B_j}{\x} - \Bjk t \; & \leq & \; 0
	\end{alignat*}
	Again, the inequality changes directions because $\Bjk < 0$.  Now consider a row $ B_j$ from $ B$:
	\[ \ip{B_j}{\x-t\e_k} =  B_j {\x} - \Bjk t \leq 0 \]
	Similarly,
	\begin{alignat*}{3}
		(\forall i\in P)\quad & t \; & \geq & \;  B_i/\Bik {\x}     & \Rightarrow \\
		(\forall i\in P)\quad & 0 \; & \geq & \;  B_i {\x} - \Bik t &
	\end{alignat*}
	Now consider a row $ B_i$ from $ B$:
	\[ \ip{B_i}{\x-t\e_k} =  B_i {\x} - \Bik t \leq 0 \]
	So, we've demonstrated that $\x-t\e_k$ satisfies all the constraints from $B$, and the left implication is shown.  So \textit{4} holds.
\end{proof}

\paragraph{Remark:}  \label{rem:fm} Proposition \ref{prop:hconeproj} highlights the properties of the matrix $B'$.  Upon close inspection, we can create a Matrix $Y$ such that $B' = YB$, and every element of $Y$ is non-negative.  Create the following set of row vectors $Y$
\begin{alignat*}{2}
	 & \e_l                  & \st & l \in Z          \\
	 & \Bik \e_j - \Bjk \e_i & \st & i\in P,\; j\in N
\end{alignat*}
Since the basis vectors simply select rows during matrix multiplication, it is clear that
\[ B' = YB \]

Now to prove:
\begin{enumerate}
	\item[\Vproj] Every coordinate-projection of an H-Cone is an H-Cone.
\end{enumerate}

\begin{proof}[proof of \Vproj]
	Here we prove the case that the coordinate projection is onto the first $d$ of $d+p$ coordinates.  Let $\set{\y\in\R^{d+\Udim}:A'\y \leq \0}$ be the H-Cone we need to project, and $\Pi$ the coordinate-projection we need to apply (the identity matrix with the last p columns deleted).  For each $1 \leq k \leq p$ we can use proposition \ref{prop:hconeproj} in an incremental manner, starting with $A'$.
	\begin{align*}
		 & \text{let } B_0 := A'                                        \\
		 & \text{for } 1 \leq k \leq p                                  \\
		 & \quad \text{let } B_k :=
		\text{result of proposition 2 applied to $B_{k-1}$, $\e_{d+k}$} \\
		 & \text{endfor}                                                \\
		 & \text{return } B_p
	\end{align*}


	Consider the resulting $B$.  Property \textit{2} holds throughout, so $B$ is finite.  After each iteration, property \textit{3} holds for $k$, so the $k$-th column is $\0$.  Since each iteration only results from non-negative combinations of the result of the previous iteration (property \textit{1}), once a column is $\0$ it remains so.  Therefore, at the end of the process, the last $p$ columns of $B$ are all $\0$.  Then, by proposition \ref{prop:hconezero}, we can apply $\Pi$ to $B$ by simply deleting the last $p$ columns of $B$.  Denote this resulting matrix $A$.  We still need to check that
	\begin{equation}\label{eq:V2equal}
		\Pi\set{\y\in\R^{d+\Udim}\st A'\y\leq\0} = \set{\x\in\R^{d}\st A\x\leq\0}
	\end{equation}
	This follows from the following:
	\begin{align}
		A'\y & \leq \0 \Rightarrow A(\Pi(\y)) \leq \0 \label{V2seteqright}                                                           \\
		A\x  & \leq \0 \Rightarrow (\exists t_1)\dots(\exists t_p) A'(\x+t_1\e_{d+1}+\cdots+t_p\e_{d+p}) \leq \0 \label{V2seteqleft}
	\end{align}
	The key observation of this verification utilizes property \textit{4} of proposition \ref{prop:hconeproj}:
	\[ (\exists t)B(\x + t\e_k) \leq \0 \Leftrightarrow B'\x \leq \0 \]
	In what follows, let $\x = \sum_{1 \leq j \leq d}x_j\e_j$.  The above property is applied sequentially to the sets $B_k$ as follows:
	\begin{alignat*}{2}
		(\exists t_p)(\exists t_{p-1})\dots(\exists t_1)                & \quad
		B_0(\x + t_1\e_{p} + t_2\e_{p-1} + \dots + t_p\e_{d}) \leq \0\; & \Leftrightarrow                                                 \\
		(\exists t_p)\dots(\exists t_2)                                 & \quad
		B_1(\x + t_2\e_{d+2} + \dots + t_p\e_{d+p})
		\leq \0                                                         & \Leftrightarrow                                                 \\
		\vdots \hspace{1.3em}                                           & \hspace{5em}\vdots                       & \vdots\hspace{0.4em} \\
		(\exists t_p)                                                   & \quad B_{p-1}(\x + t_p\e_{d+p})  \leq \0 & \Leftrightarrow      \\
		                                                                & \quad B_{p}\x  \leq \0                   &
	\end{alignat*}
	Because $A' = B_0$, and $A$ is $B_p$ with the last $p$ columns deleted, \eqref{V2seteqright} and \eqref{V2seteqleft} hold, therefore \eqref{eq:V2equal} holds, and the proof of {\Vproj} is complete, and we've shown that a coordinate projection of an H-Cone is again an H-Cone.
\end{proof}

With {\Vlift} and {\Vproj} proven, we are now certain that any V-Cone is also an H-Cone.

\section{Every H-Cone is a V-Cone}

\begin{Def}[Coordinate Hyperplane]{
		A set of the form
		\[ \set{\x \in \R^{d+\Adim} \st \ip{\x}{\e_k} = 0} =
			\set{\x \in \R^{d+\Adim} \st x_k = 0}
		\]
		is called a \em{coordinate-hyperplane}.
	}\end{Def}

We will use coordinate-hyperplanes in the following way.  We consider a V-Cone intersected with some coordinate hyperplanes, and write it in the following way:
\begin{equation}\label{eq:coneintform1}
	\set{\xv \St (\exists \t \geq 0) \xz = U'\t}
\end{equation}
If we suppose that $U' \subset \R^{d+\Adim}$, and $\Pi$ is the identity matrix with the last $\Adim$ rows deleted, then this is just a convenient way of writing:
\begin{equation}\label{eq:coneintform2}
	\Pi\big(\cone(U') \cap \set{x_{d+1} = 0}
	\cap \dots \cap \set{x_{d+\Adim} = 0}\big)
\end{equation}
The proof rests on the following three propostions:
\begin{itemize}
	\item[\Hlift] Every H-Cone is a coordinate-projection of a V-Cone intersected with some coordinate hyperplanes.
	\item[\Hint] Every V-Cone intersected with a coordinate-hyperplane is a V-Cone
	\item[\Hproj] Every coordinate-projection of a V-Cone is an V-Cone.
\end{itemize}
\begin{proof}
	Given {\Hlift}, {\Hint}, and {\Hproj}, the proof follows simply.  Given an H-Cone, we use {\Hlift} to get a description involving the coordinate-projection of a V-Cone intersected with some coordinate-hyperplanes.  We apply {\Hint} as many times as necessary to elimintate the intersections, then we can apply {\Hproj} in order to get a V-Cone.
\end{proof}

\begin{proof}[Proof of {\Hlift}]
	Let $\mA$, we now show that the H-Cone
	\[\set{\xv \st A\x \leq \0}\]
	can be written as the projection of a V-Cone intersected with some hyperplanes.  Define $U'$:
	\[ U' = \set{\zei, \eAj, \neAj, 1 \leq j \leq d,\, 1 \leq i \leq m} \]
	Note that $U'$ can be written:
	\begin{equation}\label{eq:Ulift}
		U' = \begin{pmatrix*}[r] \0 & I & -I \\ I & A & -A \end{pmatrix*}
	\end{equation}
	We then claim:
	\begin{equation}\label{eq:hliftform}
		\set{\xv \st A\x \leq \0} = \set{\xv \St (\exists \t \geq 0) \xz = U'\t}
	\end{equation}
	First, considering \eqref{eq:coneintform1} and \eqref{eq:coneintform2}, observe that this is a coordinate-projection of a V-Cone intersected with some coordinate-hyperplanes.
	Next, we note that
	\[ \xAx = \jsum x_j \eAj \]
	We can write this as a sum with all positive coefficients if we split up the $x_j$ as follows:
	\[
		\xjp = \begin{cases} x_j & x_j \geq 0 \\ 0 & x_j < 0 \end{cases} \quad\quad\quad
		\xjm = \begin{cases} 0 & x_j \geq 0 \\ -x_j & x_j < 0 \end{cases}
	\]
	Then we have
	\begin{equation} \label{eq:xAx}
		\xAx = \jsum \xjp \eAj + \jsum \xjm \neAj
	\end{equation}
	where $\xjp, \xjm \geq 0$.  Also observe that
	\[ A\x \leq \0 \Leftrightarrow (\exists \w \geq \0) \st A\x + \w = \0 \]
	This can also be written
	\begin{equation} \label{eq:Axz}
		A\x \leq \0 \Leftrightarrow (\exists \w \geq \0) \st \xAx + \zw = \xz
	\end{equation}
	\eqref{eq:xAx} and \eqref{eq:Axz} together show
	\[ A\x \leq \0 \Rightarrow (\exists \t \geq 0) \xz = U'\t \]
	Conversely, suppose
	\[ (\exists \t \geq 0) \xz = U'\t \]
	We would like to show that $A\x \leq \0$.  Let $\xjp,\xjm,w_i$ take the values of $\t$ that are coefficients of $\eAj$, $\neAj$, and $\zei$ respectively, and denote $x_j = \xjp - \xjm$.  Then we have
	\begin{align*}
		\xz & = \jsum \xjp \eAj + \jsum \xjm \neAj + \isum w_i\zei \\
		    & = \jsum x_j \eAj + \isum w_i\zei                     \\
		    & = \xAx + \zw
	\end{align*}
	where $\w \geq \0$.  By \eqref{eq:Axz} we have $A\x \leq \0$.  So \eqref{eq:hliftform} holds.
\end{proof}

The proof of {\Hint} relies upon the following proposition.
\begin{Prop}{\label{prop:Hintset}
		Let $Y \in \R^{(d+m)\times n_1}$, $1 \leq k \leq m$, and $\x$ satisfy $x_k = 0$.  Then there exists a matrix $Y' \in \R^{(d+m)\times n_2}$ with the following properties:
		\begin{enumerate}
			\item Every column of $Y'$ is a postive linear combination of columns of $Y$.
			\item $n_2$ is finite.
			\item The $k$-th row of $Y'$ is $\0$.
			\item \((\exists \t\geq\0)\x = Y\t \Leftrightarrow (\exists \t' \geq \0)\x = Y'\t'\)
		\end{enumerate}
	}\end{Prop}
%Recall that $ Y^i$ is the $i$-th column of $ Y$, and $\Yi$ is the element of $ Y$ in the $i$-th column and $k$-th row.
\begin{proof}
	We partition the columns of $ Y$:
	\begin{alignat*}{3}
		 & P &  & = i \; & \st & \Yi > 0 \\
		 & N &  & = j \; & \st & \Yj < 0 \\
		 & Z &  & = l \; & \st & \Yl = 0
	\end{alignat*}
	We then define $ Y'$:
	\[  Y' = \set{ Y^l \st l \in Z} \cup
		\set{\Yi Y^j - \Yj Y^i \st i \in P,\, j\in N} \]
	\textit{1} and \textit{2} are clear.  \textit{3} can be seen from:
	\begin{align}
		\ip{Y'^l}{\e^k}    & = 0 \nonumber                                                                \\
		\ip{Y'^{ij}}{\e^k} & = \ip{\Yi Y^j - \Yj Y^i}{\e^k} = \Yi \Yj - \Yj \Yi = 0 \label{eq:hdropZrows}
	\end{align}

	Before moving on to the proof, we first note how we may write our vectors.
	\[  Y\t = \Zsum t_k  Y^k + \Psum t_i  Y^i + \Nsum t_j  Y^j \]
	\[  Y'\t = \Zsum t_k  Y^k + \NPsum t_{ij} (\Yi Y^j - \Yj Y^i) \]
	Then, by proposition \ref{prop:closure} (cone closure), to show that the proposition is true, we need only show that, given any $t_i, t_j \geq 0$ ($t_{ij} \geq 0)$, there exists $t_{ij} \geq 0$ ($t_i, t_j \geq 0$) such that
	\begin{equation} \label{eq:coneEq}
		\Psum t_i  Y^i + \Nsum t_j  Y^j = \NPsum t_{ij} (\Yi Y^j - \Yj Y^i)
	\end{equation}
	\begin{Prop}{
			Suppose that
			\[ \Psum t_i  Y^i_{d+1} + \Nsum t_j  Y^j_{d+1} = 0\quad\quad \Yj < 0 < \Yi \]
			Then the following holds
			\begin{alignat*}{2}
				(t_i, t_j \geq 0) & \Rightarrow (\exists t_{ij} \geq 0)
				                  &                                       & \mathrm{\;such\; that\; \eqref{eq:coneEq}\; holds} \\
				(t_{ij} \geq 0)   & \Rightarrow (\exists t_i, t_j \geq 0)
				                  &                                       & \mathrm{\;such\; that\; \eqref{eq:coneEq}\; holds}
			\end{alignat*}
		}\end{Prop}
	\begin{proof}
		First note that if all $t_i = 0,t_j = 0$, then choosing $t_{ij} = 0$ satisfies \eqref{eq:coneEq}, likewise if all $t_{ij} = 0$, then $t_i = 0, t_j = 0$ satisfies \eqref{eq:coneEq}.  So suppose that some $t_i \neq 0, t_j \neq 0, t_{ij} \neq 0$.

		The right hand side of \eqref{eq:coneEq} can be written
		\[ \Nsum \left(\Psum t_{ij}\Yi\right) Y^j +
			\Psum \left(-\Nsum t_{ij}\Yj\right) Y^i \]
		This means, given $t_{ij} \geq 0$, we can choose $t_j = \Psum t_{ij}\Yi$, and $t_i = -\Nsum t_{ij}\Yj$, both of which are greater than $0$.

		Now suppose we have been given $t_i \geq 0, t_j \geq 0$.  First observe:
		\[ 0 = \Psum t_i\Yi + \Nsum t_j\Yj \Rightarrow \Psum t_i\Yi = -\Nsum t_j\Yj\]
		Denote the value in this equality as $\sigma$, and note that $\sigma > 0$.  Then
		\begin{alignat*}{3}
			\Psum t_i Y^i & = & \frac{-\Nsum t_j \Yj}{\sigma}\Psum t_i Y^i & =
			\NPsum -\frac{t_i t_j}{\sigma}\Yj Y^i                              \\
			\Nsum t_j Y^j & = & \frac{\Psum t_i \Yi}{\sigma}\Nsum t_j Y^j  & =
			\NPsum \frac{t_i t_j}{\sigma}\Yi Y^j
		\end{alignat*}
		Combining these results, we have
		\[ \Psum t_i Y^i + \Nsum t_j Y^j =
			\NPsum \frac{t_i t_j}{\sigma}(\Yi Y^j - \Yj Y^i) \]
	\end{proof}
	Finally, we can conclude that, given $\t \geq \0$, if $ Y\t$ has a $0$ in the final coordinate, then we can write it as $ Y'\t'$ where $\t' \geq \0$, and any non-negative linear combination of vectors from $Y'$ can be written as a non-negative linear combination of vetors from $Y$, and will necessarily have the $k$-th coordinate be $0$ by property \textit{3}.  So property \textit{4} holds.
\end{proof}

\begin{proof}[Proof of {\Hint}]
	In proposition \ref{prop:Hintset}, the assumption that $x_k = 0$ in property \textit{4} creates the set $\cone(Y) \cap \set{\x \st x_k = 0}$.  This set, by property \textit{4}, is $\cone(Y')$.
\end{proof}

\begin{proof}[Proof of {\Hproj}]
	We shall prove that the coordinate-projection of a V-Cone is again a V-Cone.  Let $\Pi$ be the relevant projection, then we have:
	\[ \Pi\set{U\t \st \t \geq \0} = \set{\Pi(U\t) \st \t \geq \0} =
		\set{(\Pi U)\t \st \t \geq \0} \]
	The last equality follows from associativity of matrix multiplication.  Therefore,
	\[ \Pi\big(\cone(U)\big) = \cone\big(\Pi U\big) \]
\end{proof}

\section{Reducing Polyhedra to Cones}

\begin{Def}[Hyperplane]
	Let $\y \in \R^d$, $c \in \R$.  Then a set of the form
	\[ \set{\xv \st \ip{\y}{\x} = c} \]
	is called a \em{hyperplane}.
\end{Def}

\subsection{H-Polyhedra $\leftrightarrow$ H-Cones}
We show that an H-Polyhedron can be represented as the projection of an H-Cone intersected with a hyperplane.  We begin by re-writing the expression:
\begin{equation}\label{eq:HPtoHC}
	A\x \leq \b \Leftrightarrow -\b + A\x \leq \0 \Leftrightarrow
	\big[-\b|A\big]\onex \leq \0
\end{equation}

\begin{Prop}
	Every H-Polyhedron can be written as an H-Cone intersected with the set $\set{\x\st x_0 = 1}$, and any H-Cone intersected with the set $\set{\x\st x_0 = 1}$ is an H-Polyhedron.
\end{Prop}
\begin{proof}
	We extend \eqref{eq:HPtoHC}:
	\[ \x \in \set{\xv \st A\x \leq \b} \Leftrightarrow
		\onex \in \set{\yv \st \big[-\b|A\big]\y \leq \0}\\
	\]
	We conclude, given an H-Polyhedron, we can add an extra coordinate and prepend the vector $\b$ to the left of $A$, and later we can just move this column back to the right side of the inequality and drop the extra coordinate.
\end{proof}

\subsection{V-Polyhedra $\leftrightarrow$ V-Cone}

We show that a V-Polyhedra can be reprented as a projection of a V-Cone intersected with the hyperplane $\set{\yv \st y_0 = 1}$.  Given two sets $\mV$ and $\mU$, the V-Polyhedron is given by:
\[ P_V = \set{\x + \y \st \x \in \cone(U),\, \y \in \conv(V)} \]

It isn't hard to see that
\[ \x \in P_V \Leftrightarrow \onex \in \cone\lcone \]
For the value $1$ to appear in the first coordinate, a convex combination of the vectors from $(\vec{1}, V)$ must be taken.  After that, any non-negative combination of $(\0,U)$ added to this vector won't affect the $1$ in the first coordinate.

It is more difficult to show that, given a V-Cone, that you can intersect it with the hyperplane $\set{\yv \st y_0 = 1}$ and get a V-Polytope out of it.  So let
\[ C_V = \cone(U) \cap \set{\yv \st y_0 = 1} \]
We partition $U$ into the sets:
\begin{alignat*}{3}
	 & P &  & = i & \st & \Uiz > 0 \\
	 & N &  & = j & \st & \Ujz < 0 \\
	 & Z &  & = l & \st & \Ulz = 0
\end{alignat*}
And define two new sets:
\begin{align*}
	U'  & = \set{U^l \st l \in Z} \cup \set{\Uiz U^j - \Ujz U^i \st i\in P,\, j\in N} \\
	V\; & = \set{U^i/\Uiz \st i \in P}
\end{align*}
Then I claim that
\begin{equation}\label{eq:cvtopv}
	C_V = \set{\x + \y \st \x\in\cone(U'),\, \y\in\conv(V)}
\end{equation}
Say $\x \in\cone(U')$, $\x$ can be written
\begin{align*}
	\x & = \sum_{l\in Z}t_l U^l + \NPsum t_{ij}(\Uiz U^j - \Ujz U^i)      \\
	   & = \sum_{l\in Z}t_l U^l + \Nsum \left(\Psum t_{ij}\Uiz\right) U^j
	+ \Psum \left(\Nsum -t_{ij}\Ujz\right) U^i
\end{align*}
So $\x \in \cone(U)$.  Furthermore,
\[ \ip{\e_0}{\x} = \sum_{l\in Z}t_l \Ulz + \NPsum t_{ij}(\Uiz \Ujz - \Ujz \Uiz) = 0\]
So $x_0 = 0$.  Similarly, for $\y$,
\[ \y = \Psum \lambda_i U^i/\Uiz,\quad\Psum\lambda_i = 1 \]
So $\y \in \cone(U)$, and then $\x + \y \in \cone(U)$.  Furthermore,
\[ \ip{\e_0}{\y} = \Psum \lambda_i \Uiz/\Uiz = 1\]
So $y_0 = 1$ and $x_0 + y_0 = 1$.  Then, by proposition \ref{prop:closure} (cone closure), $\x + \y \in C_V$.

Next, suppose that $\z \in C_V$, then $\z$ can be written
\[ \z = \Zsum t_l U^l + \Psum t_i U^i + \Nsum t_j U^j \]
It will be convenient to use shorter notation for these sums.  Define the following:
\begin{alignat*}{3}
	\bsigma_Z & = \Zsum t_l U^l, \quad &  & \sigma_l \, & = & \, \Zsum t_l \Ulz \;= 0 \\
	\bsigma_P & = \Psum t_i U^i, \quad &  & \sigma_i \, & = & \, \Psum t_i \Uiz       \\
	\bsigma_N & = \Nsum t_j U^j, \quad &  & \sigma_j \, & = & \, \Nsum t_j \Ujz
\end{alignat*}
Then it holds that
\[ \ip{\e_0}{\z} = \sigma_l + \sigma_i + \sigma_j = \sigma_i + \sigma_j = 1
	\quad\Rightarrow\quad -\sigma_j/\sigma_i = 1 - 1/\sigma_i \]
\[ \bsigma_P = \bsigma_P/\sigma_i + (1-1/\sigma_i)\bsigma_P
	= \bsigma_P/\sigma_i - (\sigma_j/\sigma_i)\bsigma_P \]
Using the new notation, we can rewrite $\z$:
\[ \z = \bsigma_Z + \bsigma_P + \bsigma_N
	= \bsigma_Z + \frac{\bsigma_P}{\sigma_i} - \frac{\sigma_j}{\sigma_i}\bsigma_P
	+ \frac{\sigma_i}{\sigma_i}\bsigma_N
	= \bsigma_Z + \frac{\bsigma_P}{\sigma_i} +
	\frac{\sigma_i\bsigma_N - \sigma_j\bsigma_P}{\sigma_i}
\]
Using proposition \ref{prop:closure}, we need only show that
\begin{enumerate}
	\item $\bsigma_Z \in \cone(U')$
	\item $(\sigma_i\bsigma_N - \sigma_j\bsigma_P) \in \cone(U')$
	\item $\bsigma_P/\sigma_i \in \conv(V)$
\end{enumerate}
Since each $U^l : l \in Z$ is in $C_V$, (1) holds.  We also have:
\[ \sigma_i\bsigma_N - \sigma_j\bsigma_P =
	\Psum t_i \Nsum t_j \Uiz U^j - \Nsum t_j \Psum t_i \Ujz U^i =
	\NPsum t_i t_j(\Uiz U^j - \Ujz U^i) \]
So (2) holds.  Finally,
\[ \bsigma_P/\sigma_i = \Psum t_i U^i / \sigma_i = \Psum (t_i \Uiz/\sigma_i)(U^i/\Uiz) \]
Since $\Psum (t_i \Uiz/\sigma_i) = \sigma_i / \sigma_i = 1$, it follows that $\bsigma_P / \sigma_i \in \conv(V)$.

\section{Picture of the Proof}
Here we show a diagram that represent the proof of the {\MWT}.

\newcommand{\vtohcolor}{violet}
\newcommand{\liftcolor}{blue}
\newcommand{\dropcolor}{red}
\newcommand{\dpos}{-2}
\newcommand{\mhpos}{\numexpr \dpos+1}
\newcommand{\mlpos}{\numexpr \dpos-1}
\newcommand{\PHdesc}{A\x \leq \b}
\newcommand{\CHdesc}{[-\b|A]\smallstack{x_0}{\x} \leq \vec{0}}
\newcommand{\CVpdesc}{\cone \pmb 0 & I & -I\\ I & A & -A\pme}
\newcommand{\CHpdesc}{\begin{pmatrix*}[r] \0 & -I \\ I & -U \\ -I & U \end{pmatrix*}\x \leq \0}
\newcommand{\CVdesc}{\cone\{\smallstack{\vec{0}}{U'}\cup\smallstack{\vec{1}}{V'}\}}
\newcommand{\PVdesc}{\cone(U) + \conv(V)}
\begin{figure}[H]
	\begin{centering}
		\begin{tikzpicture}[>=triangle 45]
			\node (PH) at (-1,0) {\framebox{$\PHdesc$}};
			\node (CH) at (0,\dpos) {\framebox{$\CHdesc$}};
			\node (PV) at (9,0) {\framebox{$\PVdesc$}};
			\node (CV) at (8,\dpos) {\framebox{$\CVdesc$}};
			\node (CVp) at (4,\mhpos) {\framebox{$\CVpdesc$}};
			\node (CHp) at (4,\mlpos) {\framebox{$\CHpdesc$}};
			\draw[<->,bend right=10, color=\vtohcolor]
			(PH) to (CH);
			\draw[<->,bend right=10,color=\vtohcolor]
			(CV) to (PV);
			\draw[->,bend left=15]
			(CH) to
			node[above left,color=\liftcolor]{lift}
			(CVp);
			\draw[->,bend left=15]
			(CVp) to
			node[above right,color=\dropcolor]{$\bigcap$}
			(CV);
			\draw[->,bend left=15]
			(CV) to
			node[below right,color=\liftcolor]{lift}
			(CHp);
			\draw[->,bend left=15]
			(CHp) to
			node[below left,color=\dropcolor]{$\Pi$}
			(CH);
		\end{tikzpicture}
		\caption{Diagram of the proof $P_H \leftrightarrow P_V$}
		\label{fig:h_to_v}
	\end{centering}
\end{figure}

Figure \ref{fig:h_to_v} shows the flow from an H-Polyhedron to a V-Polyhedron and back.  The \textcolor{\vtohcolor}{colored arrows} are the transformations back and forth from polyhedra to cones.  The black arrows show the transformation between cones.  V-Cones are \textcolor{\liftcolor}{lifted} to H-Cones which need to be \textcolor{\dropcolor}{projected $(\Pi)$}, and H-Cones are \textcolor{\liftcolor}{lifted} to V-Cones which need to be \textcolor{\dropcolor}{intersected $(\bigcap)$} with some coordinate-hyperplanes then projected.
