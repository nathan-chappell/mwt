%max bound
\newcommand{\MX}{10}
%draw the grid 
\newcommand{\drawGrid}{
	\clip (-1,-1) rectangle (4,4);
	\draw[style=help lines] (-\MX,-\MX) grid (\MX,\MX);
	\draw[style=very thick] (0,\MX) -- (0,-\MX);
	\draw[style=very thick] (\MX,0) -- (-\MX,0);
}
%various constants
\newcommand{\lowSlopeX}{2}
\newcommand{\lowSlopeY}{1}
\pgfmathsetmacro{\lowSlope}{\lowSlopeY / \lowSlopeX}
\newcommand{\highSlopeX}{1}
\newcommand{\highSlopeY}{2}
\pgfmathsetmacro{\highSlope}{\highSlopeY / \highSlopeX}
\newcommand{\myslope}{0}
\newcommand{\genLen}{2}
\newcommand{\lowVertX}{2}
\newcommand{\highVertX}{1.5}
\newcommand{\vertexRadius}{.08}
\newcommand{\origin}{(0,0)}
\newcommand{\spmb}{\begin{psmallmatrix*}[r]}
		\newcommand{\spme}{\end{psmallmatrix*}}
%
\tikzstyle ConeStyle=[fill, color=gray, style=semitransparent]
\tikzstyle GeneratorInd=[style=dotted, thick, color=blue]
\tikzstyle GeneratorSty=[->, style=thick, color=blue]
%
%draw the constraints: #1 - slope, #2 lt or rt
\newcommand{\getHatchLine}[3]{
	\renewcommand{\myslope}{#1}
	\pgfmathsetmacro{\myAngle}{atan{\myslope}}
	\draw[style=thick, color=blue] #3 -- +(\myAngle:-\MX);
	\draw[style=thick, color=blue] #3 -- +(\myAngle:\MX);
	\pgfmathsetmacro{\myHatchAngle}{\myAngle + #2*90}
	\foreach \x in {-\MX, -9.5, ..., \MX} {
			\draw[style=thick, color=blue] #3 ++(\myAngle:\x) -- +(\myHatchAngle:.15);
		}
}
\newcommand{\drawGenerator}[1]{
	\draw[style=GeneratorSty] \origin -- (#1:\genLen);
}
\newcommand{\drawSegment}[2]{
	\draw[fill, color=blue] #1 circle (\vertexRadius);
	\draw[fill, color=blue] #2 circle (\vertexRadius);
	\draw[style=thick, color=blue] #1 -- #2;
}
%draws the cone: #1 - low slope, #2 - high slope
\newcommand{\coneConstraints}[2]{
	\getHatchLine{#1}{1}{\origin};
	\getHatchLine{#2}{-1}{\origin};
}
%draws the cone: #1 - low slope, #2 - high slope, #3 vertex
\newcommand{\polyConstraints}[5]{
	\getHatchLine{#1}{1}{\origin};
	\getHatchLine{#2}{-1}{\origin};
	\getHatchLine{#4}{#5}{#3};
}
%draws the generators
\newcommand{\coneGenerators}[2]{
	\drawGenerator{\lowDir};
	\drawGenerator{\highDir};
	\draw[style=GeneratorInd] (\lowDir:\genLen) arc (\lowDir:\highDir:\genLen);
}
%V-Polyhedra: low, high, vert
\newcommand{\polyGenerators}[3]{
	\drawGenerator{\highDir};
	\draw[style=GeneratorInd] (\highDir:\genLen) -- ++ #3;
	\drawSegment{\origin}{\lowVert};
}
%V-Polytope
\newcommand{\vPolytope}{
	\drawSegment{\origin}{\lowVert}
	\drawSegment{\origin}{\highVert}
	\drawSegment{\lowVert}{\highVert}
}
%H-Polytope
\newcommand{\hPolytope}{
	\polyConstraints{\lowSlope}{\highSlope}{\lowVert}{-4.0}{-1}
}
%draws the h-constraints
\newcommand{\drawCone}{
	\draw[style=ConeStyle] \origin -- (\MX,\lowY) -- (\MX,\highY);
}
%draws the polyhedra
\newcommand{\drawPolyhedra}{
	\draw[style=ConeStyle] \origin -- \lowVert -- ++(\MX,\highY) -- (\MX,\highY);
}
%draws the polytope
\newcommand{\drawPolytope}{
	\draw[style=ConeStyle] \origin -- \lowVert -- \highVert;
}

%
\pgfmathsetmacro{\lowY}{\MX * \lowSlope}
\pgfmathsetmacro{\highY}{\MX * \highSlope}
\pgfmathsetmacro{\lowVertY}{\lowVertX * \lowSlope}
\pgfmathsetmacro{\highVertY}{\highVertX * \highSlope}
\pgfmathsetmacro{\lowDir}{atan{\lowSlope}}
\pgfmathsetmacro{\highDir}{atan{\highSlope}}

\newcommand{\lowVert}{(\lowVertX,\lowVertY)}
\newcommand{\highVert}{(\highVertX,\highVertY)}

\newcommand{\svec}[2]{\begin{bsmallmatrix*}[r] #1 \\ #2 \end{bsmallmatrix*}}

%H-Cone
\newcommand{\drawHCone}{
	\begin{figure}[h]
		\centering
    %\leavevmode\beginpgfgraphicnamed{jobHCone}
		\begin{tikzpicture}
			\drawGrid
			\drawCone
			\coneConstraints{\lowSlope}{\highSlope}
		\end{tikzpicture}
    %\endpgfgraphicnamed
		\caption[]{H-Cone: $\spmb
				-\highSlopeY & \highSlopeX \\
				\lowSlopeY & -\lowSlopeX
				\spme \spmb x \\ y \spme \leq \spmb 0 \\ 0 \spme$}
	\end{figure}
}

%V-Cone
\newcommand{\drawVCone}{
	\begin{figure}[h]
		\centering
    %\leavevmode\beginpgfgraphicnamed{jobVCone}
		\begin{tikzpicture}
			\drawGrid
			\drawCone
			\coneGenerators{\lowSlope}{\highSlope}
		\end{tikzpicture}
    %\endpgfgraphicnamed
		\caption[]{V-Cone: $\cone\left(
				\svec{\highSlopeX}{\highSlopeY},
				\svec{\lowSlopeX}{\lowSlopeY}\right)$}
	\end{figure}
}

%H-Polyhedra
\newcommand{\drawHPoly}{
	\begin{figure}[h]
		\centering
    %\leavevmode\beginpgfgraphicnamed{jobHPoly}
		\begin{tikzpicture}
			\drawGrid
			\drawPolyhedra
			\polyConstraints{\lowSlope}{\highSlope}{\lowVert}{\highSlope}{1}
		\end{tikzpicture}
    %\endpgfgraphicnamed
		\caption[]{H-Polyhedron: $\spmb
				-\highSlopeY & \highSlopeX \\
				\lowSlopeY & -\lowSlopeX \\
				\highSlopeY & -\highSlopeX
				\spme \spmb x \\ y \spme \leq \spmb 0 \\ 0 \\ 3 \spme$}
	\end{figure}
}

%V-Polyhedra
\newcommand{\drawVPoly}{
	\begin{figure}[h]
		\centering
    %\leavevmode\beginpgfgraphicnamed{jobVPoly}
		\begin{tikzpicture}
			\drawGrid
			\drawPolyhedra
			\polyGenerators{\lowSlope}{\highSlope}{\lowVert}
		\end{tikzpicture}
    %\endpgfgraphicnamed
		\caption[]{V-Polyhedron: $\cone\left(
				\svec{\highSlopeX}{\highSlopeY}\right) +
				\conv\left(\svec{0}{0}, \svec{\lowVertX}{1}\right)$}
	\end{figure}
}

%H-Polytope
\newcommand{\drawHPolytope}{
	\begin{figure}[h]
		\centering
    %\leavevmode\beginpgfgraphicnamed{jobHPolytope}
		\begin{tikzpicture}
			\drawGrid
			\drawPolytope
			\hPolytope
		\end{tikzpicture}
    %\endpgfgraphicnamed
		\caption[]{H-Polytope: $\spmb
				-\highSlopeY & \highSlopeX \\
				\lowSlopeY & -\lowSlopeX \\
				4 & 1
				\spme \spmb x \\ y \spme \leq \spmb 0 \\ 0 \\ 9 \spme$}
	\end{figure}
}

%V-Polytope
\newcommand{\drawVPolytope}{
	\begin{figure}[h]
		\centering
    %\leavevmode\beginpgfgraphicnamed{jobVPolytope}
		\begin{tikzpicture}
			\drawGrid
			\drawPolytope
			\vPolytope
		\end{tikzpicture}
    %\endpgfgraphicnamed
		\caption[]{V-Polytope: $\conv\left(
				\svec{0}{0},
				\svec{\lowVertX}{1},
				\svec{1.5}{3}
				\right)$}
	\end{figure}
}

%Not Full Dim
\newcommand{\drawNotFullDim}{
	\begin{figure}[h]
		\centering
    %\leavevmode\beginpgfgraphicnamed{jobNotFullDim}
		\begin{tikzpicture}
			\drawGrid
			\getHatchLine{\highSlope}{1}{\origin};
			\getHatchLine{\highSlope}{-1}{\origin};
			\getHatchLine{-1}{1}{\origin};
			\getHatchLine{0.}{1}{\origin};
		\end{tikzpicture}
    %\endpgfgraphicnamed
		\caption[]{H-Cone, not full-dimensional: \\
			$\spmb -2 & 1 \\ 2 & -1 \\ -1 & -1 \\ 0 & -1 \spme \spmb x \\ y \spme
				\leq \spmb 0 \\ 0 \\ 0 \\ 0 \spme$ }
	\end{figure}
}

%Not Pointed
\newcommand{\drawNotPointed}{
	\begin{figure}[h]
		\centering
    %\leavevmode\beginpgfgraphicnamed{jobNotPointed}
		\begin{tikzpicture}
			\clip (-3,-3) rectangle (3,3);
			\draw[style=help lines] (-\MX,-\MX) grid (\MX,\MX);
			\draw[style=very thick] (0,\MX) -- (0,-\MX);
			\draw[style=very thick] (\MX,0) -- (-\MX,0);
			\draw[style=GeneratorSty] \origin -- (135:2.);
			\draw[style=GeneratorSty] \origin -- (-45:2.);
			\draw[style=GeneratorSty] \origin -- (0,2);
			\draw[style=GeneratorSty] \origin -- (2,0);
			\draw[style=ConeStyle] \origin -- (-\MX,\MX) -- (\MX,\MX) -- (\MX,-\MX);
		\end{tikzpicture}
    %\endpgfgraphicnamed
		\caption[]{V-Cone, not pointed: $\cone\left(
				\svec{\highSlopeX}{\highSlopeY},
				\svec{\lowSlopeX}{\lowSlopeY},
				\svec{0}{2},
				\svec{2}{0}
				\right)$}
	\end{figure}
}

