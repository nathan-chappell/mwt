\chapter*{Conclusion}
\addcontentsline{toc}{chapter}{Conclusion}

In this thesis we have proven the \MWT.  The proof is constructive in nature, showing how to create the sets promised by the theorem, and a basic implementation is given.  While the proof is hardly elegant, it does show that the intuitive result of the theorem is true by brute force methods and does not require any advanced results.  Indeed, the proof is from first principles, using the language of linear algebra.

The more interesting (and, perhaps, informative) part of the paper deals with creating a testing framework for the program.  This presented an opportunity to discuss pointed and full-dimensional polyhedra, and their relation to minimal and essential representations of polyhedra.  The characteristic and dual cones were defined with some useful properties proven, invoking the Farkas Lemma in a few different forms.  As of now, the testing framework is open for extension, to include polyhedra for which minimal representations are not essentially unique.

It may be worth mentioning an anachronism presented in the text.  The chapter on testing starts off by describing some sets that are chalked up to ``wishful thinking,'' then minimality is introduced and the useful properties are derived.  In earnest, the actual progression from wishful thinking to effective testing was a bit different.  Initially, the requirements for testing were received.  Then, the first idea was to create some cones, and check if one representation was a subset of another (modulo scaling, i.e. equivalence).  This is the simplest, most immediate (and perhaps natural) answer to the challenge for testing.  After this method was decided upon, the question then arose: ``what properties of the sets representing the polyhedra are necessary to ensure the tests will work?''  Then the properties were determined.  It was only afterwards that it became clear that these properties were actually pointedness and full-dimensionality, at which point the presentation was altered to emphasize this.  Without this alteration, the presentation would be akin to: ``these seemingly arbitrary properties allow us to test the polyhedra in this manner,'' which is less pleasant to read than ``these natural classes of polyhedra have useful properties which allow us to test our implementation on them.''

It should also be mentioned that the algorithm here is not efficient.  The intermediate representations of the polyhedra may be exponential in the size of the input and output.  The ``double description'' method is a far better way to calculate the alternative representations desired, however the method is a bit more advanced and is better pursued after getting a decent grasp of the underlying problem.

The Farkas Lemma should have the last word, as it is a rather wonderful combinatorial compactification of much of the information of the \MWT.  It's main contribution here was to show that minimal sets of H-Polyhedra do exist, and then allowed us to re-use some of the work we had done with V-Polyhedra to expedite the proofs of the testing methods.
