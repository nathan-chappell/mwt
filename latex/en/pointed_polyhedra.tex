% pointed_polyhedra.tex

\chapter{Pointed and Full-Dimensional Polyhedra}

In this appendix, pointed and full dimensional polyhedra are defined, and they are shown to correspond to the polyhedra which have minimal representations.

\begin{Def}[Vertex]
	Let $P$ be a polyhedron.  A point $\v \in P$ is called a \textit{vertex} if, for any $\u \neq \0$, at least one of the following is true:
	\begin{align*}
		\v + \u \not\in P \\
		\v - \u \not\in P \\
	\end{align*}
\end{Def}

\begin{Def}[Full-Dimensional]
	Let $P \subseteq \R^d$ be a polyhedron.  $P$ is called \textit{full-dimensional} if there is a set $Z$ of are $d$ linearly independent vectors such that $Z \subseteq P$.
\end{Def}

\section{Pointed Cones}

We suppose that, given a V-Cone $\cone(U)$, that $\0\not\in U$.  This prevents dealing with useless edge cases, and the convention that $\cone(\emptyset) = \0$ prevents any questions about dealing with empty sets.  Similarly, we suppose that for any H-Cone defined by a matrix $A$, it has only non-zero rows.  The h-cone defined by an empty matrix can be taken to be the entire space.

\begin{Def}[Pointed V-Cone]
	A V-Cone $C$ is called \textit{pointed} if the origin is a vertex of $C$.
\end{Def}

\begin{Lemma}[Minimal Sets are Independent]\label{minimal_independence}
	Let $U$ be some matrix, and $\u = U\e_k$.  Then the following holds:
	\[[(\t\geq\0,\; \u = U\t) \Rightarrow \t=\e_k] \;\Rightarrow\;
		[(\t\geq\0,\;\0=U\t) \Rightarrow \t=\0]\]
\end{Lemma}

\begin{proof}
	Say $\0 = U\t$.  Then
	\[ \u = \u+\0 = U(\e_k+\t) \Rightarrow [\e_k+\t = \e_k] \Rightarrow \t = \0 \]
\end{proof}

\begin{Prop}[Minimal and Pointed V-Cones]\label{minimal_vcones}
	The pointed V-Cones are precisely the V-Cones representable by an minimal set.
\end{Prop}

\begin{Lemma}[Pointed Cones are independent]\label{pointed_independence}
	Let $C = \cone(U)$.  Then the following two statements are equivalent.
	\begin{enumerate}
		\item $C$ is pointed
		\item $(\t\geq\0,\,0=U\t)\Rightarrow \t=\0$
	\end{enumerate}
\end{Lemma}

\begin{proof}
	$(2\Rightarrow 1)$.  Suppose $C$ is not pointed, but (2) holds.  Then $\exists \u\in C\st -\u\in C$.  Then $\exists \t_1,\t_2 \geq \0$ such that $\u = U\t_1$, and $-\u = U\t_2$.  Then $\t_1,\t_2 \neq \0$, and $\t_1 + \t_2 \neq \0$, and $\0 = \u-\u = U(\t_1+\t_2)$, a contradiction.\\
	$(\neg 2\Rightarrow \neg 1)$.  Suppose that $\exists \t\geq\0\st\t\neq\0$, $U\t=\0$.  Since $\0\not\in U$, there are at least two non-zero entries of $\t$.  Suppose that $t$ is a non-zero element, $\t'$ is $\t$ with that element deleted, $\u$ is the entry of $U$ corresponding to that element, and $U'$ is $U$ with $\u$ deleted.  Then $\t',t$ are both non-negative and non-zero, and $\0 = U\t = U'\t' + \u t$.  Note that both $U'\t',\u t \in C$.  Since $U'\t' = -\u t$, $C$ is not pointed.
\end{proof}

\begin{proof}[Proof of \nameref{minimal_vcones}]
	\Cref{pointed_independence} and \Cref{minimal_independence} show that if a set does not generate a pointed cone, then it is not minimal.

	Next let $C = \cone(U)$ be pointed, where $U$ is not minimal.  That is,
	\[(\exists \u\in U)\; \u=U\e_i,\, \u = U\t,\, \t\neq\e_i \]
	We show that we can remove $\u$ from $U$ without affecting the $C$.  We then conclude that we can continue to reduce $U$ until it is minimal.\\
	Let $U' = U\setminus\set{\u}$, so that $\u = U'\t' + t\e_i$ where $\t'\geq\0$, $t\geq 0$.  We consider two cases:
	\begin{align*}
		(0\leq t<1) & \quad (1-t)\u = U'\t' \Rightarrow \u = U'\t'/(1-t) \Rightarrow \u \in \cone(U') \\
		(1 \leq t)  & \quad U(\t'+(t-1)\e_i)=\0 \Rightarrow \t'+(t-1)\e_i=\0 \Rightarrow \t'=\0,t=1
	\end{align*}
	That $C$ is pointed is used twice in the second line.  The result of the second line is that $\t=\e_i$, a contradiction.
\end{proof}

\section{Full-Dimensional Cones}

\begin{Prop}[Minimal and Pointed H-Cones]\label{minimal_hcones}
	The full-dimensional H-Cones are precisely the H-Cones representable by an minimal set.
\end{Prop}

\begin{Lemma}[Full-Dimensional Cones are independent]\label{full_independence}
	Let $C = \HC{A}$.  Then the following two statements are equivalent.
	\begin{enumerate}
		\item $C$ is full-dimensional
		\item $(\t\geq\0,\,\t^T A=\0)\Rightarrow \t=\0$
	\end{enumerate}
\end{Lemma}

\begin{proof}
	$(1\Rightarrow 2)$.  Let $V$ be a linearly-independent set spanning the whole space contained in $C$, that is $AV\leq\0$.  Because $A$ has only non-zero rows, and $V$ is linearly-independent, for any row $A_i$ of $A$ it holds that $\ip{A_i}{V} \neq 0$.  Since $\t\geq\0$ and $AV\leq\0$:
	\[ \t^T(AV) = \0 \;\Rightarrow\; (\forall i,j)\,(AV)_i^j \neq 0 \Rightarrow t_i = 0\]
	Since every row of $AV$ contains a non-zero element, $\t=\0$.\\
	$(\neg 1\Rightarrow\neg 2)$.  If $C$ is not full-dimensional, then there is a non-trivial orthogonal compliment $O$ to the span of $C$.  Let $\z\in O$, $\z\neq\0$, then $A\x\leq\0 \Rightarrow \z^T\x = 0$.  Then there is a $\y_1\geq\0$ such that $\y_1^T A = \z$.  If there were not, then, by \nameref{farkas_lemma}:
	\[ (\exists \x)\;A\x\leq\0, \x^T\z > 0 \]
	which contradicts $\z\in O$.  Since $-\z\in O$, there is also a $\y_2\geq\0$ such that $\y_2^T A = -\z$.  Since $\z\neq\0$, and $\y_1,\y_2\geq\0$, $\y_1+\y_2\neq\0$, and $(\y_1+\y_2)^T A = \z-\z = \0$.
\end{proof}

\begin{proof}[Proof of \nameref{minimal_hcones}]
	\Cref{full_independence} and \Cref{minimal_independence} show that if a set does not generate a full-dimensional cone, then it is not minimal.\\
	We show that, given a full dimensional H-Cone $\HC{A}$, if $A$ is not minimal, then we can remove a row, implying that $A$ can be reduced to an minimal set.  Suppose that $\exists\t\geq\0$, $A_i = \t^T A$, and $\t\neq\e_i$.  Let $A'$ be $A$ with the $i$-th row deleted, and $A_i = \t'^T A' + tA_i$.  Then
	\begin{align}
		 & \0 = \t'^T A' + (t-1)A_i \label{rxhc1} \\
		 & (1-t)A_i = \t'^T A' \label{rxhc2}
	\end{align}
	We need to show that $A\x\leq\0 \Leftrightarrow A'\x\leq\0$.  The right implication is clear.  If we can show that $(\exists\y\geq\0)\;\y^T A'=A_i$, then the left implication will be shown.  If there is no such $\y$, then, by \nameref{farkas_lemma}:
	\[(\exists\x)\; A'\x\leq\0,\, \x^T A_i > 0 \]
	Suppose that $t \geq 1$, then by \eqref{rxhc1} $\0 = (\t'+(t-1)\e_i)^T A$.  By \nameref{full_independence}, $\t'+(t-1)\e_i = \0$, so $\t'=\0, t=1$, and $\t = \e_i$, a contradiction.  Otherwise, $0\leq t < 1$.  Then by \eqref{rxhc2}:
	\[\t'^T A' = (1-t)A_i \quad\quad \t'^T A'\x \leq 0 < (1-t)A_i\x \]
	Which is another contradiction, so we see that $A'\x\leq\0 \Leftrightarrow A\x\leq\0$.
\end{proof}

\section{Pointed V-Polyhedra}

Now we show that V-Polyhedra that have an minimal representation are precisely the pointed polyhedra.

\begin{Lemma}[Reducing Convex Hull]\label{reduce_conv}
	Suppose that $\v_i = V\e_i$, and there is some convex combinator $\blambda \neq \e_i$ such that $\v_i = V\blambda$.  Then
	\[ \conv(V) = \conv(V\setminus\set{\v_i}) \]
\end{Lemma}

\begin{proof}
	We show that $\v\in\conv(V\setminus\set{\v_i})$.  The result follows from \cref{conv_conv}.  If $\lambda_i = 0$ then there is nothing to be done, and $\lambda_i \neq 1$ because $\blambda \neq \e_i$ by assumption.  Otherwise, define: $\blambda'$ given by:
	\[ \lambda'_j =
		\begin{cases} 0 & j = i \\ \lambda_j/(1-\lambda_i) & j \neq i \end{cases} \]
	It follows that all $\lambda'_j \geq 0$.  Furthermore:
	\begin{align}
		 & \sum_j \lambda'_j = \frac{\sum_{j\neq i} \lambda_j}{1-\lambda_i}
		= \frac{1 - \lambda_i}{1-\lambda_i} = 1 \label{minconv_conv}                \\
		 & \sum_j \lambda'_j\v_j = \frac{\sum_{j\neq i} \lambda_j\v_j}{1-\lambda_i}
		= \frac{\v_i - \lambda_i\v_i}{1-\lambda_i} = \v_i \label{minconv_inc}
	\end{align}
	\eqref{minconv_conv} shows that $\blambda'$ is a convex combinator, which then shows that \eqref{minconv_inc} implies ${\v\in\conv(V\setminus\set{\v_i})}$.
\end{proof}

\begin{Prop}[Minimal Convex Hulls]\label{min_conv}
  Let $V$ be a set that satisfies the following: for any $v\in V$, and any convex combinator $\blambda$,
  \begin{equation} V\blambda = \v \Rightarrow \blambda = \e_i \label{eq:conv_min}\end{equation}
  Where $\v=V\e_i$.  Then every $\v\in V$ is a vertex of $V$.
\end{Prop}

\begin{proof}
  Say, for some $\v\in V$, there is a $\u,\,-\u$ such that $\v+\u\in V,\, \v-\u\in V$.  Then there are convex combinators $\blambda_1,\blambda_2$ such that $\v+\u=V\blambda_1$, and $\v-\u=\blambda_2$.  Then $\blambda = \blambda_1/2 + \blambda_2/2$ is another convex combinator, and $\blambda \neq \e_i$.  But 
    \[\v = \frac{\v+\u}{2} + \frac{\v-\u}{2} = V\blambda_1/2 + V\blambda_2/2 = V\blambda \]
    which contradicts \eqref{eq:conv_min}.
\end{proof}

\begin{Prop}[Minimal and Pointed V-Polyhedra]
	The pointed V-Polyhedra are precisely those representable by a minimal pair.
\end{Prop}

\begin{proof}
	Let $C=\VP{U}{V}$, where $(U,V)$ is a minimal pair.  That is, they satisfy:
	\begin{alignat*}{1}
		 & \t \geq \0,\; \u = U\t \Rightarrow \t = \e_k                              \\
		 & \t \geq \0, \blambda \geq \0, \ip{\blambda}{\1} = 1, \v = U\t + V\blambda
		\Rightarrow \t = \0, \blambda = \e_l
	\end{alignat*}
	We show that $C$ has a vertex.  If $V$ is empty, then this is shown in \nameref{minimal_vcones}.  So assume that $V$ not empty.  Suppose that there is some $\v=V\e_i$ and $\u\neq\0$ such that $\v-\u\in C$, $\v+\u\in C$.  Then $\v-\u=U\t_1+V\blambda_1$ and $\v+\u=U\t_2+V\blambda_2$ for some $\t_{1,2}\geq \0$ and convex combinators $\blambda_{1,2}$.  It follows that
  \[\v = \frac{\v-\u}{2}+\frac{\v+\u}{2} = \frac{U\t_1+V\blambda_1}{2} + \frac{U\t_2+V\blambda_2}{2}
       = U\frac{\t_1+\t_2}{2}+V\frac{\blambda_1+\blambda_2}{2} \]
  By the assumption of minimality, $(\t_1+\t_2)/2 = 0$, so $\t_1=\t_2=0$, and $(\blambda_1 + \blambda_2)/2 = \e_i$, so $\blambda_1=\blambda_2=\e_i$.  But then $\u=0$, a contradiction.

  Next we show how to take a pointed V-Polyhedron given by some $U$ and $V$, and reduce it to a minimal pair.  First, reduce $U$ as done before for V-Cones.  Next, suppose there is some $\v\in V$ such that $\v=U\t + V\blambda$ where $\t\neq\0$.  Let $V'=V\setminus\{\v\}\cup\{V\blambda\}$.  It is easy to see that $\v\in\VP{U}{V'}$, so do this for all such $\v$.  Then, reduce $V'$ as in \nameref{reduce_conv}.  What is left is a pair $(U',V')$ such that $\VP{U}{V} = \VP{U'}{V'}$, and $(U',V')$ is a minimal pair.
\end{proof}

\section{Full-Dimensional H-Polyhedra}

