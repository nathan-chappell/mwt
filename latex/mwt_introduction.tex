%introduction.tex

% importance
The \MWT{} is a theorem of utmost importance in the theory of polyhedra.  Ziegler refers to it as the ``main theorem.''  It is a theorem which is easy to intuitively understand, but is surprisingly difficult to prove.  Going through the proof in detail, however, is a good excersice because it introduces various other imporant notions and techniques, as well as helps to mentally clarify the basic definitions relevant to the theory.

Polyhedra are ``finitely generated'' convex subsets of $\R^n$.  ``Finitely generated'' here means that they can be described in a finite manner, in particular as a convex hull of either: a finite set of points and ray, or a finite intersection of halfspaces.  The \MWT{} states that these two finite representations are ``equivalent,'' that every polyhedra can be described either way.

For a more concrete example, consider a polygon in the plane \todo{provide some nice pictures here}.

% paper overview
In this paper, the theorem and relvant definitions will be stated, and a proof will be given.  The proof will closely follow Ziegler's, however it is the author's intent to make this paper as accessible as possible \todo{who is the target audience?}.  Therefore, some concepts will be stated in somewhat more painstaking detail than some may find necessary, however these details will have to be addressed to implement an enumeration-algorithm that will accompany the proof.

% diagram
Perhaps the most important part of the paper is the diagram in figure \ref{fig:proof}.  This diagram illustrates the main steps of the proof, and will serve as a ``roadmap'' to the implementation.

