\documentclass[fleqn]{article}
\usepackage{amsmath,amsfonts,amsthm}
\usepackage{mathtools, bm}
\usepackage{listings, color}

\newcommand{\lstVecMat}{\lstinputlisting[firstline=10, firstnumber=10, lastline=11]{../cpp/include/common.h}}
\newcommand{\lstVPoly}{\lstinputlisting[firstline=13, firstnumber=13, lastline=16]{../cpp/include/common.h}}
\newcommand{\lstD}{\lstinputlisting[firstline=20, firstnumber=20, lastline=20]{../cpp/include/common.h}}
\newcommand{\lstIstream}{\lstinputlisting[firstline=22, firstnumber=22, lastline=25]{../cpp/include/common.h}}
\newcommand{\lstOstream}{\lstinputlisting[firstline=26, firstnumber=26, lastline=29]{../cpp/include/common.h}}
\newcommand{\lstInputError}{\lstinputlisting[firstline=30, firstnumber=30, lastline=30]{../cpp/include/common.h}}
\newcommand{\lstUsage}{\lstinputlisting[firstline=36, firstnumber=36, lastline=36]{../cpp/include/common.h}}
\newcommand{\lstTranspose}{\lstinputlisting[firstline=41, firstnumber=41, lastline=41]{../cpp/include/common.h}}
\newcommand{\lstProjectM}{\lstinputlisting[firstline=44, firstnumber=44, lastline=44]{../cpp/include/common.h}}
\newcommand{\lstCheckEmpty}{\lstinputlisting[firstline=129, firstnumber=129, lastline=131]{../cpp/src/common.cpp}}
\newcommand{\lstFME}{\lstinputlisting[firstline=166, firstnumber=166, lastline=183]{../cpp/src/common.cpp}}
\newcommand{\lstFMEPart}{\lstinputlisting[firstline=169, firstnumber=169, lastline=172]{../cpp/src/common.cpp}}
\newcommand{\lstFMEMove}{\lstinputlisting[firstline=174, firstnumber=174, lastline=174]{../cpp/src/common.cpp}}
\newcommand{\lstFMEConvolute}{\lstinputlisting[firstline=176, firstnumber=176, lastline=181]{../cpp/src/common.cpp}}
\newcommand{\lstLiftHcone}{\lstinputlisting[firstline=13, firstnumber=13, lastline=13]{../cpp/include/hcone.h}}
\newcommand{\lstIntersectVCone}{\lstinputlisting[firstline=53, firstnumber=53, lastline=59]{../cpp/src/hcone.cpp}}
\newcommand{\lstHconeToVcone}{\lstinputlisting[firstline=63, firstnumber=63, lastline=69]{../cpp/src/hcone.cpp}}
\newcommand{\lstLiftVcone}{\lstinputlisting[firstline=9, firstnumber=9, lastline=9]{../cpp/include/vcone.h}}
\newcommand{\lstProjectHCone}{\lstinputlisting[firstline=51, firstnumber=51, lastline=57]{../cpp/src/vcone.cpp}}
\newcommand{\lstVconeToHcone}{\lstinputlisting[firstline=61, firstnumber=61, lastline=67]{../cpp/src/vcone.cpp}}
\newcommand{\lstProjectZero}{\lstinputlisting[firstline=12, firstnumber=12, lastline=16]{../cpp/src/polyhedra.cpp}}
\newcommand{\lstNormalizeP}{\lstinputlisting[firstline=20, firstnumber=20, lastline=30]{../cpp/src/polyhedra.cpp}}
\newcommand{\lstHpolyToHCone}{\lstinputlisting[firstline=34, firstnumber=34, lastline=42]{../cpp/src/polyhedra.cpp}}
\newcommand{\lstHconeToHPoly}{\lstinputlisting[firstline=46, firstnumber=46, lastline=54]{../cpp/src/polyhedra.cpp}}
\newcommand{\lstVpolyToVCone}{\lstinputlisting[firstline=59, firstnumber=59, lastline=74]{../cpp/src/polyhedra.cpp}}
\newcommand{\lstVconeToVPoly}{\lstinputlisting[firstline=78, firstnumber=78, lastline=92]{../cpp/src/polyhedra.cpp}}
\newcommand{\lstHpolyToVPoly}{\lstinputlisting[firstline=96, firstnumber=96, lastline=98]{../cpp/src/polyhedra.cpp}}
\newcommand{\lstVpolyToHPoly}{\lstinputlisting[firstline=100, firstnumber=100, lastline=102]{../cpp/src/polyhedra.cpp}}


\renewcommand{\vec}[1]{\mathbf{#1}}
\newcommand{\set}[1]{\left\{#1\right\}}
\DeclareMathOperator{\cone}{cone}
\DeclareMathOperator{\conv}{conv}
\newcommand{\ip}[2]{\left\langle #1, #2 \right\rangle}

%special letters
\newcommand{\R}{\mathbb{R}}
\newcommand{\0}{\vec{0}}
\newcommand{\x}{\vec{x}}
\newcommand{\y}{\vec{y}}
\newcommand{\z}{\vec{z}}
\newcommand{\e}{\vec{e}}
\newcommand{\w}{\vec{w}}
\renewcommand{\t}{\vec{t}}
\renewcommand{\v}{\vec{v}}
\renewcommand{\b}{\vec{b}}
\newcommand{\faij}{\forall i\in P,\forall j \in N}

%symbols
\newcommand{\st}{\;|\;}
\newcommand{\St}{\;\Big|\;}

%constants
\newcommand{\Udim}{p}
\newcommand{\Vdim}{n}
\newcommand{\Adim}{m}
\newcommand{\mspaceA}{\R^{{\Adim}\times d}}
\newcommand{\mspaceB}{\R^{m_1\times (d+\Udim)}}
\newcommand{\mspaceC}{\R^{m_2\times (d+\Udim)}}

%matrices and vectors with domain
\newcommand{\bv}{\b \in \R^{\Adim}}
\newcommand{\tv}{\t \in \R^{\Udim}}
\renewcommand{\l}{\bm{\lambda}}
\newcommand{\lv}{\l \in \R^{\Vdim}}
\newcommand{\yv}{\y \in \R^{d+1}}
\newcommand{\xv}{\x \in \R^d}
\newcommand{\mV}{V \in \R^{d\times \Vdim}}
\newcommand{\mU}{U \in \R^{d\times \Udim}}
\newcommand{\mA}{A \in \mspaceA}
\newcommand{\mB}{B \in \mspaceB}
\newcommand{\mC}{B' \in \mspaceC}
\newcommand{\xt}{\begin{pmatrix*}\x\\ \t\end{pmatrix*}}
\newcommand{\xw}{\begin{pmatrix*}\x\\ \w\end{pmatrix*}}
\newcommand{\xAx}{\begin{pmatrix*}\x\\ A\x\end{pmatrix*}}
\newcommand{\xz}{\begin{pmatrix*}\x\\ \0\end{pmatrix*}}
\newcommand{\xx}{\begin{pmatrix*}x_0\\ \x\end{pmatrix*}}
\newcommand{\onex}{\begin{pmatrix*} 1\\ \x\end{pmatrix*}}
\newcommand{\zw}{\begin{pmatrix*}\0\\ \w\end{pmatrix*}}
\newcommand{\eAj}{\begin{pmatrix*}\e_j\\ A^j\end{pmatrix*}}
\newcommand{\neAj}{\begin{pmatrix*}[r]-\e_j\\ -A^j\end{pmatrix*}}
\newcommand{\ee}{\begin{pmatrix*} \0 \\ 1 \end{pmatrix*}}
\newcommand{\zei}{\begin{pmatrix*} \0 \\ \e_i \end{pmatrix*}}
\newcommand{\lcone}{\begin{pmatrix*} \0 & \vec{1} \\ U & V \end{pmatrix*}}
\newcommand{\xjp}{x_j^+}
\newcommand{\xjm}{x_j^-}
\newcommand{\Yi}{Y^i_{k}}
\newcommand{\Yj}{Y^j_{k}}
\newcommand{\Yl}{Y^l_{k}}
\newcommand{\Uiz}{U^i_{0}}
\newcommand{\Ujz}{U^j_{0}}
\newcommand{\Ulz}{U^l_{0}}

%sums
\newcommand{\tusum}{\sum_{1\leq j \leq \Udim}t_j U^j}
\newcommand{\lvsum}{\sum_{1\leq j \leq \Vdim}\lambda_j V^j}
\newcommand{\lsum}{\sum_{1\leq j \leq \Vdim}\lambda_j}
\newcommand{\jsum}{\sum_{1\leq j \leq d}}
\newcommand{\isum}{\sum_{1\leq i \leq n}}
\newcommand{\Psum}{\sum_{i\in P}}
\newcommand{\Nsum}{\sum_{j\in N}}
\newcommand{\Zsum}{\sum_{l\in Z}}
\newcommand{\NPsum}{\sum_{\substack{i\in P \\ j\in N}}}
\newcommand{\isconv}{\lambda_j \geq 0 \lsum = 1}
\newcommand{\sumi}{\sum\nolimits_i}

\newtheorem{Def}{Definition}
\newtheorem{Thm}{Theorem}
\newtheorem{Prop}{Proposition}

%text macros
\newcommand{\MWT}{Minkowski-Weyl Theorem }

\begin{document}

\section{\MWT}

\begin{Def}[non-negative linear combination]{
  Let $\mU$, $\tv, \t \geq \0$, then \( \tusum = U\t\) is called a \textbf{non-negative linear combination} of $U$.
}\end{Def}

\begin{Def}[V-Cone]{
  Let $\mU$.  The set of all non-negative linear combinations of $U$ is denoted $\cone(U)$.  Such a set is called a \textbf{V-Cone}.
}\end{Def} 

\begin{Def}[convex combination]{
  Let $\mV$, $\lv, \l \geq\0, \lsum = 1$, then \( \lvsum \) is called a \textbf{convex combination} of V.  The set of all convex combinations of $V$ is denoted $\conv(V)$.
}\end{Def}

\begin{Def}[V-Polyhedron]{
  Let $\mV$, $\mU$.  Then the set
  %\[ \set{\tusum + \lvsum\St\,t_j \geq 0,\isconv} \]
  \[ \set{\x + \y \st \x \in \cone(U),\, \y \in \conv(V)} \]
  is called a \em{V-Polyhedron}.
}\end{Def}

\begin{Def}[H-Polyhedron]{
  Let $\mA$, $\bv$.  Then the set
  \[ \set{\xv \St A\x \leq \b} \]
  is called an \em{H-Polyhedron}.
}\end{Def}

\begin{Def}[H-Cone]{
  Let $\mA$. Then the set
  \[ \set{\xv \St A\x \leq \0} \]
  is called an \em{H-Cone}.
}\end{Def}

The following theorem is the basic result to be proved in this thesis, which states that V-Polyhedra and H-Polyhedra are two different representations of the same objects.

\begin{Thm}[\MWT]{
  Every V-Polyhedron is an H-Polyhedron, and every H-Polyhedron is a V-Polyhedron.
}\end{Thm}

The proof proceeds by first showing that V-Cones are representable as H-Cones, and H-Cones are representable as V-Cones.  Then it is shown that the case of polyhedra can be reduced to cones.

\begin{Thm}[\MWT for Cones]{
  Every V-Cone is an H-Cone, and every H-Cone is a V-Cone.
}\end{Thm}

A simple but useful property of cones is that they are closed under addition and positive scaling.

\begin{Prop}\label{prop:closure}
  Let $C$ be either an H-Cone or a V-Cone, for each $i$ $\x^i \in C$, and $c_i \geq 0$.  Then:
  \[ \sumi c_i \x^i \in C \]
\end{Prop}

\begin{proof}
  First we prove Proposition \ref{prop:closure} for H-Cones, then for V-Cones.
  If, for each $i$, $A\x^i \leq \0$, then $A(c_i\x^i) = t_iA\x^i \leq \0$, and
  \[ A\left(\sumi c_i\x^i\right) = \sumi A (c_i\x^i) = 
            \sumi c_i A\x^i \leq \sumi \0 \leq \0 \]
  So, $\sumi c_i\x^i \in C$ when $C$ is an H-Cone.  Next, suppose that $C = \cone(U)$, and for each $i$, $\exists \t_i \geq \0: \x^i = U\t_i$.  Then $c_i\t_i \geq \0$, and $\sumi c_i\t_i \geq \0$.  Therefore
  \[ \sumi c_i\x^i = \sumi c_i U\t_i = \sumi U(c_i\t_i) 
                   = U\left(\sumi c_i\t_i\right) \]
  So, $\sumi c_i\x^i \in C$ when $C$ is a V-Cone.
\end{proof}

This proposition will be used in the following way: if we wish to show that $\sumi c_i\x^i$ in a member of some cone $C$, it suffices to show that, for each $i$, $c_i \geq 0$ and $\x^i \in C$.

\newcommand{\Vcomp}{(V1)}
\newcommand{\Vproj}{(V2)}

\section{Every V-Cone is an H-Cone}

\begin{Def}[Coordinate Projection]{
  Let $I$ be the identity matrix.  Then the matrix $I'$ formed by deleting some rows from $I$ is called a \textbf{coordinate-projection}.
}\end{Def}

  The proof rests on the following two propostions:
  \begin{itemize}
  \item[\Vcomp] Every V-Cone is a coordinate-projections of an H-Cone.
  \item[\Vproj] Every coordinate-projection of an H-Cone is an H-Cone.
  \end{itemize}
\begin{proof}
  Given {\Vcomp} and {\Vproj}, the proof follows simply.  Given a V-Cone, we use {\Vcomp}, to get a description involving coordinate-projection of an H-Cone.  Then we can apply {\Vproj} in order to get an H-Cone.
\end{proof}

\begin{proof}[Proof of {\Vcomp}]
  We prove that every V-Cone is a coordinate-projection of an H-Cone, by giving an explicit formula.  Let ${\mU}$, and observe that
  \[ \cone(U) = \set{U\t \st \t \in \R^{\Udim},\, \t \geq \0} = 
    \set{\xv \st (\exists \tv)\,\x = U\t,\, \t \geq \0} \]
  We will collect $\t$ and $\x$ on the left side of the inequality, treating $\t$ as a variable and expressing its contraints as linear inequalities, then project away the coordinates corresponding to $\t$.  The following expression takes one step:
  \begin{equation}\label{eq:tleqz}
  \t \geq \0 \Leftrightarrow -I\t \leq \0
  \end{equation}
  And using the equality: $a = 0 \Leftrightarrow a \leq 0 \land -a \leq 0$, and block matrix notation, we take the second step.
  \begin{equation}\label{eq:xeqt}
   \x = U\t \Leftrightarrow \x - U\t = \0 \Leftrightarrow
    \begin{pmatrix*}[r] I & -U \\ -I & U \end{pmatrix*} \xt \leq \0
  \end{equation}
  Comparing \eqref{eq:tleqz} and \eqref{eq:xeqt}, we define a new matrix $A' \in \R^{(\Udim+2d)\times(d+\Udim)}$:
  \[A' = \begin{pmatrix*}[r] \0 & -I \\ I & -U \\ -I & U \end{pmatrix*} \]
  then we can rewrite $\cone(U)$:
  \begin{equation*}
     \cone(U) = \set{ \xv \St A'\xt \leq \0}
  \end{equation*}
  Let $\Pi \in \set{0,1}^{d\times(d+\Udim)}$ be the identity matrix in $\R^{(d+\Udim)\times(d+\Udim)}$, but with the last $\Udim$-rows deleted.  Then $\Pi$ is a coordinate projection, and the above expression can be written:
  \begin{equation}\label{eq:vconelift}
    \cone(U) = \Pi\left(\set{ \y \in \R^{d+\Udim} \st A'\y \leq \0}\right)
  \end{equation}
  This is a coordinate projection of an H-Cone, and {\Vcomp} is shown.
\end{proof}
To prove {\Vproj}, we use two separate propositions.
\begin{Prop}{\label{prop:hconezero}
  Let $B\in\R^{m'\times(d+\Udim)}$, $B'$ be $B$ with the last $\Udim$ columns deleted, and $\Pi$ the identity matrix with the last $\Udim$ rows deleted (i.e. $B' = \Pi B$).  Furthermore, suppose that the last $\Udim$ columns of $B$ are $\0$.  Then
  \[ \Pi\left(\set{\y \in \R^{d+\Udim} \st B\y \leq \0}\right) = 
              \set{\x\in\R^d\st B'\x\leq\0} \]
}\end{Prop}
\begin{proof}
  Recall that $B\y \leq \0$ means that $(\forall i)\ip{B_i}{\y} \leq 0$.  By the way we've defined $B$, any row $B_i$ of $B$ can be written $(B'_i,\0)$, with $\0 \in \R^\Udim$.  Rewriting $\y\in\R^{d+\Udim}$ as $(\x,\w)$ with $\x\in\R^d,\w\in\R^\Udim$, so that $\x = \Pi(\y)$.  Then
  \[ \ip{B}{\y} = \ip{(B'_i,\0)}{(\x,\w)} = \ip{B'_i}{\x} = \ip{B'_i}{\Pi(\y)} \]
  It follows that
  \[ \ip{B_i}{\y} \leq 0 \Leftrightarrow \ip{B'_i}{\Pi(\y)} \leq 0 \]
  Since $B_i$ is an arbitrary row of $B$, the proposition is shown.
\end{proof}

In order to use the above proposition, we need a matrix with $\0$ columns.  The next proposition shows us how to do so, one column at a time.

\begin{Prop}{\label{prop:hconeproj}
Let $\mB$, $1 \leq k \leq p$, and $\x = \sum_{i\neq k}x_i \e_i$.  Then there exists a matrix $\mC$ with the following properties:
  \begin{enumerate}
    \item Every row of $B'$ is a postive linear combination of rows of $B$.
    \item $m_2$ is finite.
    \item The $k$-th column of $B'$ is $\0$.
    \item \((\exists t)B(\x + t\e_k) \leq \0 \Leftrightarrow B'\x \leq \0\)
  \end{enumerate}
}\end{Prop}
\newcommand{\Bik}{B^k_i}
\newcommand{\Bjk}{B^k_j}
\newcommand{\Blk}{B^k_l}
\begin{proof}
  Partition the rows of $B$ as follows:
  \begin{alignat*}{3}
  &P &&= i &\st &\Bik > 0 \\
  &N &&= j &\st &\Bjk < 0 \\
  &Z &&= l &\st &\Blk = 0
  \end{alignat*}
  Then let $B'$ be a matrix with rows of the following forms:
  \begin{alignat*}{3}
    &C_l    &&= B_l &\st &l \in Z \\
    &C_{ij} &&= \Bik B_j - \Bjk B_i &\st &i \in P, j \in N
  \end{alignat*}
  \textit{1} and \textit{2} are clear.  \textit{3} can be seen from:
  \[ \ip{C_l}{\e_k} = 0 \]
  \begin{equation}\label{eq:vdropZrows}
  \ip{C_{ij}}{\e_k} = \ip{\Bik B_j - \Bjk B_i}{\e_k} = \Bik \Bjk - \Bjk \Bik = 0
  \end{equation}
  The right direction of \textit{4} is shown in the following calculations.  Because $\Blk = 0$:
  \[ \ip{B_l}{\x+t\e_k} = \ip{B_l}{\x} + t\Blk = \ip{B_l}{\x} = \ip{C_l}{\x} \]
  So:
  \[ \ip{B_l}{\x+t\e_k} \leq 0 \Rightarrow \ip{C_l}{\x} \leq 0 \]
  For rows indexed by $P,N$, we observe \eqref{eq:vdropZrows}, and have:
  \[ \ip{\Bik B_j - \Bjk B_i}{\x + t\e_k} = \ip{\Bik B_j - \Bjk B_i}{\x} \]
  Now, we use property \textit{1}:
  \[ \ip{B_i}{\x+t\e_k} \leq 0,\; \ip{B_j}{\x+t\e_k} \leq 0 \Rightarrow 
     \ip{\Bik B_j - \Bjk B_i}{\x + t\e_k} \leq 0\]
  Therefore 
  \[ \ip{\Bik B_j - \Bjk B_i}{\x} \leq 0 \]
     
  Now suppose that $B'\x \leq \0$.  The task is to find a $t$ so that $B\x \leq \0$.  Looking at \eqref{eq:vdropZrows}, any choice of $t$ we make will be okay for rows indexed by $Z$.  So the task is to find a $t$ so that the inequality holds for rows indexed by $P$ and $N$.  Observe
\begin{align*}
  \faij&\quad \ip{\Bik B_j - \Bjk B_i}{\x} \leq 0 \Leftrightarrow
\end{align*}
\vspace{-2.5em}
\begin{alignat*}{4}
  \faij&\quad \ip{\Bik B_j}{\x} &&\leq\; \ip{\Bjk B_i}{\x} &\Leftrightarrow \\
  \faij&\quad \ip{B_j/\Bjk}{\x} &&\geq\; \ip{B_i/\Bik}{\x} &\;\Leftrightarrow 
\end{alignat*}
\vspace{-2em}
\begin{align*}
   \quad\min_{j\in N}  \ip{B_j/\Bjk}{\x} \,\geq\, \max_{i\in P} \ip{B_i/\Bik}{\x}
\end{align*}
Note that the third inequality changes directions because $\Bjk < 0$.  Now we choose $t$ to lie in this last interval, and show that we can use it to satisfy all of the constraints given by $ B$.  So, we have a $t$ such that
\[ \min_{j\in N}  \ip{B_j/\Bjk}{\x} \geq t 
          \geq \max_{i\in P} \ip{B_i/\Bik}{\x} \]
In particular,
\begin{alignat*}{2}
 (\forall j\in N)\quad & \ip{B_j/\Bjk}{\x}     \;&\geq&\; t \Rightarrow \\
 (\forall j\in N)\quad & \ip{B_j}{\x} - \Bjk t \;&\leq&\; 0
\end{alignat*}
Again, the inequality changes directions because $\Bjk < 0$.  Now consider a row $ B_j$ from $ B$:
\[ \ip{B_j}{\x-t\e_k} =  B_j {\x} - \Bjk t \leq 0 \]
Similarly,
\begin{alignat*}{3}
 (\forall i\in P)\quad & t \;&\geq&\;  B_i/\Bik {\x} &\Rightarrow \\
 (\forall i\in P)\quad & 0 \;&\geq&\;  B_i {\x} - \Bik t &
\end{alignat*}
Now consider a row $ B_i$ from $ B$:
\[ \ip{B_i}{\x-t\e_k} =  B_i {\x} - \Bik t \leq 0 \]
So, we've demonstrated that $\x-t\e_k$ satisfies all the constraints from $B$, and the left implication is shown.  So \textit{4} holds.
\end{proof}
Now to prove:
\begin{enumerate}
  \item[\Vproj] Every coordinate-projection of an H-Cone is an H-Cone.
\end{enumerate}

\begin{proof}[proof of \Vproj]
  Here we prove the case that the coordinate projection is onto the first $d$ of $d+p$ coordinates.  Let $\set{\y\in\R^{d+\Udim}:A'\y \leq \0}$ be the H-Cone we need to project, and $\Pi$ the coordinate-projection we need to apply (the identity matrix with the last p columns deleted).  For each $1 \leq k \leq p$ we can use proposition \ref{prop:hconeproj} in an incremental manner, starting with $A'$.
\begin{align*}
  &\text{let } B_0 := A'\\
  &\text{for } 1 \leq k \leq p \\
  &\quad \text{let } B_k := 
         \text{result of proposition 2 applied to $B_{k-1}$, $\e_{d+k}$} \\
  &\text{endfor} \\
  &\text{return } B_p
\end{align*}


Consider the resulting $B$.  Property \textit{2} holds throughout, so $B$ is finite.  After each iteration, property \textit{3} holds for $k$, so the $k$-th column is $\0$.  Since each iteration only results from non-negative combinations of the result of the previous iteration (property \textit{1}), once a column is $\0$ it remains so.  Therefore, at the end of the process, the last $p$ columns of $B$ are all $\0$.  Then, by proposition \ref{prop:hconezero}, we can apply $\Pi$ to $B$ by simply deleting the last $p$ columns of $B$.  Denote this resulting matrix $A$.  We still need to check:
\begin{equation}\label{V2seteqright}
   A'\y \leq \0 \Leftrightarrow A(\Pi(\y)) \leq \0 
\end{equation}
\begin{equation}\label{V2seteqleft}
  (\exists t_1)\dots(\exists t_p) A'(\x+t_1\e_{d+1}+\cdots+t_p\e_{d+p}) \leq \0
          \Leftrightarrow A\x \leq \0
\end{equation}
Then, using \eqref{V2seteqright} and \eqref{V2seteqleft}, it is easy to see that:
\begin{equation}\label{eq:V2equal}
   \Pi\set{\y\in\R^{d+\Udim}\st A'\y\leq\0} = \set{\x\in\R^{d}\st A\x\leq\0} 
\end{equation}
The key observation of this verification utilizes property \textit{4} of proposition \ref{prop:hconeproj}:
  \[ (\exists t)B(\x + t\e_k) \leq \0 \Leftrightarrow B'\x \leq \0 \]
In what follows, let $\x = \sum_{1 \leq j \leq d}x_j\e_j$.  The above property is applied sequentially to the sets $B_k$ as follows:
\begin{alignat*}{2}
  (\exists t_p)(\exists t_{p-1})\dots(\exists t_1)&\quad 
                B_0(\x + t_1\e_{p} + t_2\e_{p-1} + \dots + t_p\e_{d}) \leq \0\;&
                   \Leftrightarrow \\
               (\exists t_p)\dots(\exists t_2)&\quad 
                B_1(\x + t_2\e_{d+2} + \dots + t_p\e_{d+p}) 
                    \leq \0 &\Leftrightarrow \\
           \vdots \hspace{1.3em} & \hspace{5em}\vdots & \vdots\hspace{0.4em} \\
   (\exists t_p)&\quad B_{p-1}(\x + t_p\e_{d+p})  \leq \0 & \Leftrightarrow \\
           &\quad B_{p}\x  \leq \0 & \Leftrightarrow
\end{alignat*}
Because $A' = B_0$, and $A$ is $B_p$ with the last $p$ columns deleted, \eqref{V2seteqright} and \eqref{V2seteqleft} hold, therefore \eqref{eq:V2equal} holds, and the proof of {\Vproj} is complete, and we've shown that a coordinate projection of an H-Cone is again an H-Cone.
\end{proof}

With {\Vcomp} and {\Vproj} proven, we are now certain that any V-Cone is also an H-Cone.

\section{Every H-Cone is a V-Cone}

\begin{Def}[Coordinate Hyperplane]{
  A set of the form
  \[ \set{\x \in \R^{d+\Adim} \st \ip{\x}{\e_k} = 0} = 
     \set{\x \in \R^{d+\Adim} \st x_k = 0}
  \]
  is called a \em{coordinate-hyperplane}.
}\end{Def}

We will use coordinate-hyperplanes in the following way.  We consider a V-Cone intersected with some coordinate hyperplanes, and write it in the following way:
\begin{equation}\label{eq:coneintform1}
   \set{\xv \St (\exists \t \geq 0) \xz = U'\t}
\end{equation}
If we suppose that $U' \subset \R^{d+\Adim}$, and $\Pi$ is the identity matrix with the last $\Adim$ rows deleted, then this is just a convenient way of writing:
\begin{equation}\label{eq:coneintform2}
  \Pi\big(\cone(U') \cap \set{x_{d+1} = 0} 
                        \cap \dots \cap \set{x_{d+\Adim} = 0}\big)
\end{equation}
\newcommand{\Hlift}{\textit{H1}}
\newcommand{\Hint}{\textit{H2}}
\newcommand{\Hproj}{\textit{H3}}
  The proof rests on the following three propostions:
  \begin{itemize}
  \item[\Hlift] Every H-Cone is a coordinate-projection of a V-Cone intersected with some coordinate hyperplanes.
  \item[\Hint] Every V-Cone intersected with a coordinate-hyperplane is a V-Cone
  \item[\Hproj] Every coordinate-projection of a V-Cone is an V-Cone.
  \end{itemize}
\begin{proof}
  Given {\Hlift}, {\Hint}, and {\Hproj}, the proof follows simply.  Given an H-Cone, we use {\Hlift} to get a description involving the coordinate-projection of a V-Cone intersected with some coordinate-hyperplanes.  We apply {\Hint} as many times as necessary to elimintate the intersections, then we can apply {\Hproj} in order to get a V-Cone.
\end{proof}

\begin{proof}[Proof of {\Hlift}]
Let $\mA$, we now show that the H-Cone 
  \[\set{\xv \st A\x \leq \0}\]
can be written as the projection of a V-Cone intersected with some hyperplanes.  Define $U'$:
  \[ U' = \set{\eAj, \neAj, \zei, 1 \leq j \leq d,\, 1 \leq i \leq m} \]
  Then we claim:
\begin{equation}\label{eq:hliftform}
   \set{\xv \st A\x \leq \0} = \set{\xv \St (\exists \t \geq 0) \xz = U'\t}
\end{equation}
  First, considering \eqref{eq:coneintform1} and \eqref{eq:coneintform2}, observe that this is a coordinate-projection of a V-Cone intersected with some coordinate-hyperplanes.
Next, we note that
  \[ \xAx = \jsum x_j \eAj \]
We can write this as a sum with all positive coefficients if we split up the $x_j$ as follows:
\[
   \xjp = \begin{cases} x_j & x_j \geq 0 \\ 0 & x_j < 0 \end{cases} \quad\quad\quad
   \xjm = \begin{cases} 0 & x_j \geq 0 \\ -x_j & x_j < 0 \end{cases}
\]
Then we have
\begin{equation} \label{eq:xAx}
  \xAx = \jsum \xjp \eAj + \jsum \xjm \neAj
\end{equation}
where $\xjp, \xjm \geq 0$.  Also observe that
  \[ A\x \leq \0 \Leftrightarrow (\exists \w \geq \0) \st A\x + \w = \0 \]
This can also be written
\begin{equation} \label{eq:Axz}
  A\x \leq \0 \Leftrightarrow (\exists \w \geq \0) \st \xAx + \zw = \xz
\end{equation}
\eqref{eq:xAx} and \eqref{eq:Axz} together show
\[ A\x \leq \0 \Rightarrow (\exists \t \geq 0) \xz = U'\t \]
Conversely, suppose
\[ (\exists \t \geq 0) \xz = U'\t \]
We would like to show that $A\x \leq \0$.  Let $\xjp,\xjm,w_i$ take the values of $\t$ that are coefficients of $\eAj$, $\neAj$, and $\zei$ respectively, and denote $x_j = \xjp - \xjm$.  Then we have
\begin{align*} 
\xz &= \jsum \xjp \eAj + \jsum \xjm \neAj + \isum w_i\zei\\
    &= \jsum x_j \eAj + \isum w_i\zei \\
    &= \xAx + \zw
\end{align*}
where $\w \geq \0$.  By \eqref{eq:Axz} we have $A\x \leq \0$.  So \eqref{eq:hliftform} holds.
\end{proof}

The proof of {\Hint} relies upon the following proposition.
\begin{Prop}{\label{prop:Hintset}
Let $Y \in \R^{(d+m)\times n_1}$, $1 \leq k \leq m$, and $\x$ satisfy $x_k = 0$.  Then there exists a matrix $Y' \in \R^{(d+m)\times n_2}$ with the following properties:
  \begin{enumerate}
    \item Every column of $Y'$ is a postive linear combination of rows of $B$.
    \item $n_2$ is finite.
    \item The $k$-th row of $Y'$ is $\0$.
    \item \((\exists \t\geq\0)\x = Y\t \Leftrightarrow (\exists \t' \geq \0)\x = Y'\t'\)
  \end{enumerate}
}\end{Prop}
Recall that $ Y^i$ is the $i$-th column of $ Y$, and $\Yi$ is the element of $ Y$ in the $i$-th column and $k$-th row.  
\begin{proof}
We partition the columns of $ Y$:
  \begin{alignat*}{3}
  &P &&= i \;&\st &\Yi > 0 \\
  &N &&= j \;&\st &\Yj < 0 \\
  &Z &&= l \;&\st &\Yl = 0
  \end{alignat*}
We then define $ Y'$:
\[  Y' = \set{ Y^l \st l \in Z} \cup 
          \set{\Yi Y^j - \Yj Y^i \st i \in P,\, j\in N} \]
\textit{1} and \textit{2} are clear.  \textit{3} can be seen from:
  \[ \ip{Y'^l}{\e^k} = 0 \]
  \begin{equation}\label{eq:vdropZrows}
  \ip{Y'^{ij}}{\e^k} = \ip{\Yi Y^j - \Yj Y^i}{\e^k} = \Yi \Yj - \Yj \Yi = 0
  \end{equation}

Before moving on to the proof, we first note how to write our vectors.
\[  Y\t = \Zsum t_k  Y^k + \Psum t_i  Y^i + \Nsum t_j  Y^j \]
\[  Y'\t = \Zsum t_k  Y^k + \NPsum t_{ij} (\Yi Y^j - \Yj Y^i) \]
Then, by proposition \ref{prop:closure}, to show that the proposition is true, we need only show that, given any $t_i, t_j \geq 0$ ($t_{ij} \geq 0)$, there exists $t_{ij} \geq 0$ ($t_i, t_j \geq 0$) such that
\begin{equation} \label{eq:coneEq}
  \Psum t_i  Y^i + \Nsum t_j  Y^j = \NPsum t_{ij} (\Yi Y^j - \Yj Y^i)
\end{equation}
\begin{Prop}{
  Suppose that 
  \[ \Psum t_i  Y^i_{d+1} + \Nsum t_j  Y^j_{d+1} = 0\quad\quad \Yj < 0 < \Yi \]
  Then the following holds
\begin{alignat*}{2} 
(t_i, t_j \geq 0)& \Rightarrow (\exists t_{ij} \geq 0)
                       && \mathrm{\;such\; that\; \eqref{eq:coneEq}\; holds}  \\
(t_{ij} \geq 0)& \Rightarrow (\exists t_i, t_j \geq 0)
                       && \mathrm{\;such\; that\; \eqref{eq:coneEq}\; holds}
\end{alignat*}
}\end{Prop}
\begin{proof}
First note that if all $t_i = 0,t_j = 0$, then choosing $t_{ij} = 0$ satisfies \eqref{eq:coneEq}, likewise if all $t_{ij} = 0$, then $t_i = 0, t_j = 0$ satisfies \eqref{eq:coneEq}.  So suppose that some $t_i \neq 0, t_j \neq 0, t_{ij} \neq 0$.

The right hand side of \eqref{eq:coneEq} can be written
\[ \Nsum \left(\Psum t_{ij}\Yi\right) Y^j + 
   \Psum \left(-\Nsum t_{ij}\Yj\right) Y^i \]
This means, given $t_{ij} \geq 0$, we can choose $t_j = \Psum t_{ij}\Yi$, and $t_i = -\Nsum t_{ij}\Yj$, both of which are greater than $0$.

Now suppose we have been given $t_i \geq 0, t_j \geq 0$.  First observe:
\[ 0 = \Psum t_i\Yi + \Nsum t_j\Yj \Rightarrow \Psum t_i\Yi = -\Nsum t_j\Yj\]
Denote the value in this equality as $\sigma$, and note that $\sigma > 0$.  Then
\begin{alignat*}{3} 
\Psum t_i Y^i &= &\frac{-\Nsum t_j \Yj}{\sigma}\Psum t_i Y^i &= 
                     \NPsum -\frac{t_i t_j}{\sigma}\Yj Y^i \\
\Nsum t_j Y^j &= &\frac{\Psum t_i \Yi}{\sigma}\Nsum t_j Y^j &= 
                     \NPsum \frac{t_i t_j}{\sigma}\Yi Y^j
\end{alignat*}
Combining these results, we have
\[ \Psum t_i Y^i + \Nsum t_j Y^j = 
                     \NPsum \frac{t_i t_j}{\sigma}(\Yi Y^j - \Yj Y^i) \]
\end{proof}
Finally, we can conclude that, given $\t \geq \0$, if $ Y\t$ has a $0$ in the final coordinate, then we can write it as $ Y'\t'$ where $\t' \geq \0$, and any non-negative linear combination of vectors from $Y'$ can be written as a non-negative linear combination of vetors from $Y$, and will necessarily have the $k$-th coordinate be $0$ by property \textit{3}.  So property \textit{4} holds.
\end{proof}

\begin{proof}[Proof of {\Hint}]
In proposition \ref{prop:Hintset}, the assumption that $x_k = 0$ in property \textit{4} creates the set $\cone(Y) \cap \set{\x \st x_k = 0}$.  This set, by property \textit{4}, is $\cone(Y')$.
\end{proof}

\begin{proof}[Proof of {\Hproj}]
  We shall prove that the coordinate-projection of a V-Cone is again a V-Cone.  Let $\Pi$ be the relevant projection, then we have:
  \[ \Pi\set{U\t \st \t \geq \0} = \set{\Pi(U\t) \st \t \geq \0} = 
        \set{\Pi(U)\t \st \t \geq \0} \]
The last equality follows from associativity of matrix multiplication.  Therefore,
  \[ \Pi\big(\cone(U)\big) = \cone\big(\Pi(U)\big) \]
\end{proof}

\section{Reducing Polyhedra to Cones}

\begin{Def}[Hyperplane]
  Let $\y \in \R^d$, $c \in \R$.  Then a set of the form
  \[ \set{\xv \st \ip{\y}{\x} = c} \]
  is called a \em{hyperplane}.
\end{Def}

\subsection{H-Polyhedra $\leftrightarrow$ H-Cones}
We show that an H-Polyhedron can be represented as the projection of an H-Cone intersected with a hyperplane.  We begin by re-writing the expression:
\begin{equation}\label{eq:HPtoHC}
  A\x \leq \b \Leftrightarrow -\b + A\x \leq \0 \Leftrightarrow
     \big[-\b|A\big]\onex \leq \0 
\end{equation}

\begin{Prop}
  Every H-Polyhedron can be written as an H-Cone intersected with the set $\set{\x\st x_0 = 1}$, and any H-Cone intersected with the set $\set{\x\st x_0 = 1}$ is an H-Polyhedron.
\end{Prop}
\begin{proof}
We extend \eqref{eq:HPtoHC}:
\[ \x \in \set{\xv \st A\x \leq \b} \Leftrightarrow
   \onex \in \set{\yv \st \big[-\b|A\big]\y \leq \0}\\
\]
We conclude, given an H-Polyhedron, we can add an extra coordinate and prepend the vector $\b$ to the left of $A$, and later we can just move this column back to the right side of the inequality and drop the extra coordinate.
\end{proof}

\subsection{V-Polyhedra $\leftrightarrow$ V-Cone}

We show that a V-Polyhedra can be reprented as a projection of a V-Cone intersected with the hyperplane $\set{\yv \st y_0 = 1}$.  Given two sets $\mV$ and $\mU$, the V-Polyhedron is given by:
  \[ P_V = \set{\x + \y \st \x \in \cone(U),\, \y \in \conv(V)} \]

It isn't hard to see that
  \[ \x \in P_V \Leftrightarrow \onex \in \cone\lcone \]
For the value $1$ to appear in the first coordinate, a convex combination of the vectors from $(\vec{1}, V)$ must be taken.  After that, any non-negative combination of $(\0,U)$ added to this vector won't affect the $1$ in the first coordinate.

It is more difficult to show that, given a V-Cone, that you can intersect it with the hyperplane $\set{\yv \st y_0 = 1}$ and get a V-Polytope out of it.  So let
  \[ C_V = \cone(U) \cap \set{\yv \st y_0 = 1} \]
We partition $U$ into the sets:
  \begin{alignat*}{3}
  &P &&= i &\st &\Uiz > 0 \\
  &N &&= j &\st &\Ujz < 0 \\
  &Z &&= l &\st &\Ulz = 0
  \end{alignat*}
And define two new sets:
\begin{align*} 
  U'  &= \set{U^l \st l \in Z} \cup \set{\Uiz U^j - \Ujz U^i \st i\in P,\, j\in N} \\
  V\; &= \set{U^i/\Uiz \st i \in P}
\end{align*}
Then I claim that
\begin{equation}\label{eq:cvtopv}
  C_V = \set{\x + \y \st \x\in\cone(U'),\, \y\in\conv(V)}
\end{equation}
Say $\x \in\cone(U')$, $\x$ can be written
\begin{align*}
\x &= \sum_{l\in Z}t_l U^l + \NPsum t_{ij}(\Uiz U^j - \Ujz U^i) \\
   &= \sum_{l\in Z}t_l U^l + \Nsum \left(\Psum t_{ij}\Uiz\right) U^j
                             + \Psum \left(\Nsum -t_{ij}\Ujz\right) U^i
\end{align*}
So $\x \in \cone(U)$.  Furthermore,
\[ \ip{\e_0}{\x} = \sum_{l\in Z}t_l \Ulz + \NPsum t_{ij}(\Uiz \Ujz - \Ujz \Uiz) = 0\]
So $x_0 = 0$.  Similarly, for $\y$,
\[ \y = \Psum \lambda_i U^i/\Uiz,\quad\Psum\lambda_i = 1 \]
So $\y \in \cone(U)$, and then $\x + \y \in \cone(U)$.  Furthermore,
\[ \ip{\e_0}{\y} = \Psum \lambda_i \Uiz/\Uiz = 1\]
So $y_0 = 1$ and $x_0 + y_0 = 1$.  Then, by proposition \ref{prop:closure}, $\x + \y \in C_V$.

Next, suppose that $\z \in C_V$, then $\z$ can be written
\[ \z = \Zsum t_l U^l + \Psum t_i U^i + \Nsum t_j U^j \]
It will be convenient to use shorter notation for these sums.  Define the following:
\begin{alignat*}{3}
  \sigma_Z &= \Zsum t_l U^l, \quad &&\sigma_l \,&=&\, \Zsum t_l \Ulz \;= 0\\
  \sigma_P &= \Psum t_i U^i, \quad &&\sigma_i \,&=&\, \Psum t_i \Uiz \\
  \sigma_N &= \Nsum t_j U^j, \quad &&\sigma_j \,&=&\, \Nsum t_j \Ujz
\end{alignat*}
Then it holds that
\[ \ip{\e_0}{\z} = \sigma_l + \sigma_i + \sigma_j = \sigma_i + \sigma_j = 1 
      \quad\Rightarrow\quad -\sigma_j/\sigma_i = 1 - 1/\sigma_i \]
\[ \sigma_P = \sigma_P/\sigma_i + (1-1/\sigma_i)\sigma_P 
            = \sigma_P/\sigma_i - (\sigma_j/\sigma_i)\sigma_P \]
Using the new notation, we can rewrite $\z$:
\[ \z = \sigma_Z + \sigma_P + \sigma_N 
      = \sigma_Z + \frac{\sigma_P}{\sigma_i} - \frac{\sigma_j}{\sigma_i}\sigma_P
                 + \frac{\sigma_i}{\sigma_i}\sigma_N 
      = \sigma_Z + \frac{\sigma_P}{\sigma_i} + 
                   \frac{\sigma_i\sigma_N - \sigma_j\sigma_P}{\sigma_i}
\]
Using proposition \ref{prop:closure}, we need only show that 
\begin{enumerate}
  \item $\sigma_Z \in \cone(U')$
  \item $(\sigma_i\sigma_N - \sigma_j\sigma_P)/\sigma_i \in \cone(U')$ 
  \item $\sigma_P/\sigma_i \in \conv(V)$
\end{enumerate}
Since each $U^l : l \in Z$ is in $C_V$, (1) holds.  We also have:
\[ \sigma_i\sigma_N - \sigma_j\sigma_P = 
    \Psum t_i \Nsum t_j \Uiz U^j - \Nsum t_j \Psum t_i \Ujz U^i = 
    \NPsum t_i t_j(\Uiz U^j - \Ujz U^i) \]
So (2) holds.  Finally,
\[ \sigma_P/\sigma_i = \Psum t_i U^i / \sigma_i = \Psum (t_i \Uiz/\sigma_i)(U^i/\Uiz) \]
Since $\Psum (t_i \Uiz/\sigma_i) = \sigma_i / \sigma_i = 1$, it follows that $\sigma_P / \sigma_i \in \conv(V)$.

\section{C++ Implementation}
The above transformation have been implemented in C++.  There are four programs:
\begin{itemize}
	\item \texttt{hcone\_to\_vcone.cpp}
	\item \texttt{hpoly\_to\_vpoly.cpp}
	\item \texttt{vcone\_to\_hcone.cpp}
	\item \texttt{vpoly\_to\_hpoly.cpp}
\end{itemize}
They all run the indicated transformations.  They read the description of the object from standard input, and write the result to standard output.  They take no arguments, however if any arguments are passed then they display the following usage information, which also includes the input format:
\begin{verbatim}
usage: (vcone_to_hcone | vpoly_to_hpoly | hcone_to_vcone |hpoly_to_vpoly)
input is read on stdin,transformed object written on stdout
input format is as follows:
 
hcone, vcone:= dimension   (vector)*
       hpoly:= dimension+1 (vector constraint)*
       vpoly:= dimension   (vector)* 'U' (vector)*

dimension   is a positive integer
vector      is (dimension) doubles separated by whitespace
constraint  is a double (the value b_i in <A_i,x> <= b_i)
hvector     is (dimension) doubles separated by whitespace
'U'         is the literal character 'U'
\end{verbatim}

The files pertaining to the implementation will be discussed in the following subsections, but here is a table showing the include dependencies followed by a short summary of the files. \\

\begin{tabular}{|l|p{9em}|}
\hline
file & includes \\
\hline
\texttt{common.cpp} &  \texttt{common.h} \\
\texttt{hcone.cpp} &  \texttt{hcone.h} \\
\texttt{polyhedra.cpp} &   \texttt{hcone.h}\hfill \texttt{polyhedra.h} \hfill\hspace{2em} \texttt{vcone.h} \\
\texttt{vcone.cpp} &  \texttt{vcone.h} \\
\hline
\texttt{hcone.h} &  \texttt{common.h} \\
\texttt{polyhedra.h} &  \texttt{common.h} \\
\texttt{vcone.h} &  \texttt{common.h} \\
\hline
\end{tabular}\\

\vspace{1em}

Here is a very brief summary of the files mentioned in the above table, more details are given in subsequent sections.

\begin{itemize}
  \item \texttt{common.\{cpp,h\}}\\
    Types, IO, and Fourier Motzkin elimination.
  \item \texttt{hcone.\{cpp,h\}}\\
    Functions to transform H-Cone $\to$ V-Cone.
  \item \texttt{polyhedra.\{cpp,h\}}\\
    Transforms between polytopes and polyhedra.
  \item \texttt{vcone.\{cpp,h\}}\\
    Functions to transform V-Cone $\to$ H-Cone.
\end{itemize}

%Since all interesting implementation details are located in the \texttt{*.cpp} files, only these files will be examined below.

\lstset{
  language=C++,
  backgroundcolor=\color[rgb]{.9,.9,.9},
  basicstyle=\small\tt,
  keywordstyle=\color[rgb]{0,.2,1},
  commentstyle=\color[rgb]{0,.6,.2},
  numbers=left
}

\subsection{\texttt{include/common.h}, \texttt{src/common.cpp}}
\lstVecMat
The types \lstinline{Vector} and \lstinline{Matrix} are used for representing the polyhedra.  The \lstinline{std::valarray} template is used because it has built-in vector-space operations (sum and scaling).  \lstinline{std::vector}, is used, however other sequence containers could be used.

\lstVPoly
Because a V-Polyhedron needs two matrices two represent it, a the simple struct \lstinline{VPoly} is defined.

\lstD
\lstinline{d} is a global variable denoting ``dimension'' used by some operations (i.e. reading vectors and projections).  \lstinline{transpose} is used in Fourier-Motzkin elimination when creating the alternate representations.  \lstinline{check_empty_matrix}  returns true if there are either no \lstinline{Vector}s, or the first \lstinline{Vector} is empty.

\lstIstream
\lstOstream
The stream input and output \lstinline{operator>>} and \lstinline{operator<<} are defined to handle input and output as described in \textit{usage}.  

\lstInputError
The input operators may throw an exception of type \lstinline{input_error} if the input dimension is not positive, or if there is an invalid number of values following the dimension.

\lstUsage
Outputs the usage message above.

\lstCheckEmpty
\lstinline{check_empty_matrix} checks for the corner case of \lstinline{Matrix} operations.  It return \lstinline{true} if either the \lstinline{Matrix} has no rows (columns) or the first row (column) is empty.

\lstTranspose
\lstinline{transpose} transposes the matrix.

\lstProjectM
\lstinline{project_matrix} is used to take only only the first \lstinline{d} entries of each vector in the \lstinline{Matrix}.

\lstFourierMotzkin
\lstinline{fourier_motzkin} takes a \lstinline{Matrix M} and a coordinate \lstinline{k} and creates the set which either corresponds to a projection of an H-Cone (without actually doing the projection), or the intersection of a V-Cone with a coordinate-hyperplane.

\lstFourierMotzkinPart
Partitions \lstinline{M} into logical sets $Z,P,N$ that satisfy the following:\\

\begin{tabular}{|l|l|l|}
  \hline
set & range & property \\
  \hline
  $Z$ & [\lstinline|M.begin()|, \lstinline|z_end| $)$ & 
      \lstinline|it| $\in Z \Leftrightarrow$ \lstinline|(*it)[k]| $ = 0$ \\
  \hline
  $P$ & [\lstinline|z_end|, \lstinline|p_end| ) & 
      \lstinline|it| $\in P \Leftrightarrow$ \lstinline|(*it)[k]| $ > 0$ \\
  \hline
  $N$ & [\lstinline|p_end|, \lstinline|M.end()|) & 
      \lstinline|it| $\in N \Leftrightarrow$ \lstinline|(*it)[k]| $ < 0$ \\
  \hline
\end{tabular}\\

\lstFourierMotzkinMove
Moves $Z$ into the result.

\lstFourierMotzkinConvolute
Convolutes the vectors in the way described in Propositions \ref{prop:hconeproj} and \ref{prop:Hintset} (concerning projecting an H-Cone and intersecting a V-Cone with a coordinate-hyperplane), and push them into the result \lstinline{Matrix}.  In particular, it creates the sets which correspond to
\[ \Bik B_j - \Bjk B_i \st i \in P,\, j \in N \\ \]

\subsection{\texttt{include/hcone.h}, \texttt{src/hcone.cpp}}
\lstinputlisting{../cpp/include/hcone.h}
\texttt{hcone.h} and \texttt{hcone.cpp} implement the transformation from H-Cone to V-Cone.

\lstLiftHcone
Takes a \texttt{Matrix} representing an H-Cone and creates the new matrix
  \[ \set{\eAj, \neAj, \zei, 1 \leq j \leq d,\, 1 \leq i \leq m} \]
where $A$ represents \texttt{hcone}.  This operation is justified by {\Hlift}.

\lstIntersectVCone
Here the tools from \texttt{common.h} are used to implement the algorithm described in proposition \ref{prop:Hintset}, where a V-Cone is sequentially intersected with coordinate-hyperplanes.  The result of these intersections is the projected to the original space.  These operations are justified by {\Hint} and {\Hproj}.

\lstHconeToVcone
This function does a sanity check and then return the transformed \texttt{hcone}.\\

\subsection{\texttt{src/polyhedra.cpp}}
\subsection{\texttt{src/vcone.cpp}}

\end{document}
